\magnification=1150
\hsize 7.0truein
\hoffset -0.3in
\outer\def\begsubsect#1\par{\vskip0pt plus.15\vsize\penalty-150
 \vskip0pt plus-.15\vsize\medskip\vskip\parskip
 \leftline{\it#1}\nobreak\smallskip\indent}
\def\uncatcodespecials{\def\do##1{\catcode`##1=12 }\dospecials}
\def\leaderfill{\leaders\hbox to 1em{\hss.\hss}\hfill}
\def\doverbatim#1{\def\next##1#1{##1\endgroup}\next}
\def\setupverbatim{\tt 
  \def\par{\leavevmode\endgraf} \catcode`\`=\active
  \obeylines \uncatcodespecials \obeyspaces}
  {\obeyspaces\global\let =\ }
\def\listing#1{\par\begingroup\setupverbatim\input#1 \endgroup}
\def\verbatim{\begingroup\setupverbatim\doverbatim}
\parskip 1.7ex plus0.5ex minus0.5ex
\centerline{\bf SKYCALC USER'S MANUAL}
\medskip
\centerline{John Thorstensen, Dept. Physics and Astronomy, Dartmouth College}
\medskip
\centerline{\it Table of Contents}
\medskip
\line{\qquad 0. Introduction -- Why Bother? \leaderfill 3 \qquad}
\smallskip
\line{\qquad 1. The Interactive Almanac \leaderfill 3 \qquad}
\line{\qquad\qquad Overview \leaderfill 3 \qquad}
\line{\qquad\qquad 1.1 Basic Use of the Program \leaderfill 5 \qquad}
\line{\qquad\qquad\qquad Starting up \leaderfill 5 \qquad}
\line{\qquad\qquad\qquad Specifying a date ({\tt y)} and Getting an Almanac ({\tt a}) \leaderfill 6 \qquad}
\line{\qquad\qquad\qquad Specifying RA and dec ({\tt r} and {\tt d}) and tabulating hourly airmass ({\tt h}) \leaderfill 7 {\it ff} \qquad}
\line{\qquad\qquad\qquad A Word about Hard Copy and Log Files \leaderfill 9 \qquad}
\line{\qquad\qquad\qquad Specifying time ({\tt t}) and getting Instantaneous Circumstances with {\tt =} \leaderfill 10 {\it ff} \qquad}
\line{\qquad\qquad\qquad Observability through a Season ({\tt o}) \leaderfill 12 \qquad}
\line{\qquad\qquad\qquad Looking at Current Parameters with {\tt l} \leaderfill 13 \qquad}
\line{\qquad\qquad 1.2 More Commands \leaderfill 15 \qquad}
\line{\qquad\qquad\qquad Quitting the program ({\tt Q}) \leaderfill 15 \qquad}
\line{\qquad\qquad\qquad Printing a Menu -- ({\tt ?}) \leaderfill 15 \qquad}
\line{\qquad\qquad\qquad A Word about `eXtra Goodies' ({\tt x}) \leaderfill 16 \qquad}
\line{\qquad\qquad\qquad Setting the Time to Now -- {\tt T} \leaderfill 16 \qquad}
\line{\qquad\qquad\qquad Enabling/Disabling Automatic Clock Update ({\tt xU})
\leaderfill 16 \qquad}
\line{\qquad\qquad\qquad Changing the site ({\tt s}) \leaderfill 17 \qquad}
\line{\qquad\qquad\qquad UT time input and `night dates' ({\tt g} and {\tt n}) \leaderfill 18 \qquad}
\line{\qquad\qquad\qquad Coordinate epoch ({\tt e}), batch precession ({\tt xb}), apparent place ({\tt xa}) \leaderfill 18 \qquad}
\line{\qquad\qquad\qquad Proper Motions ({\tt p}) \leaderfill 18 \qquad}
\line{\qquad\qquad\qquad Coordinate conversions ({\tt xc}) \leaderfill 19 \qquad}
\line{\qquad\qquad\qquad Julian Date calculations and setting ({\tt xj} and {\tt xJ}) \leaderfill 19 -- 20 \qquad}
\line{\qquad\qquad\qquad $\rm TDT - UT$ calculation ({\tt xd}) \leaderfill 20 \qquad}
\line{\qquad\qquad\qquad Major planets ({\tt m}) \leaderfill 20 \qquad}
\line{\qquad\qquad\qquad Setting to the Zenith  ({\tt xZ}) \leaderfill 21 \qquad}
\line{\qquad\qquad\qquad Object list handling  -- {\tt xR, xl, xN,} and {\tt xS} ({\tt m}) \leaderfill 21{\it ff} \qquad}
\line{\qquad\qquad\qquad Predicting periodic phenomena -- {\tt xf} and {\tt xv} \leaderfill 23{\it ff} \qquad}
\line{\qquad\qquad\qquad {\tt xp} -- Parallax factors and aberration \leaderfill 23 \qquad}
\line{\qquad\qquad\qquad {\tt xy} -- Day of year \leaderfill 24 \qquad}
\line{\qquad\qquad 1.3 Algorithms, Accuracy and Limitations \leaderfill 25 \qquad}
\line{\qquad\qquad\qquad Calendars and times \leaderfill 25 \qquad}
\line{\qquad\qquad\qquad Sun and Moon \leaderfill 26 \qquad}
\line{\qquad\qquad\qquad The Major Planets \leaderfill 28 \qquad}
\line{\qquad\qquad\qquad Geographical Limitations \leaderfill 29 \qquad}
\line{\qquad\qquad\qquad Precession and Apparent Place \leaderfill 29 \qquad}
\line{\qquad\qquad\qquad Local Mean Sidereal Time \leaderfill 30 \qquad}
\line{\qquad\qquad\qquad Parallactic Angle \leaderfill 30 \qquad}
\line{\qquad\qquad\qquad Airmass \leaderfill 30 \qquad}
\line{\qquad\qquad\qquad Barycentric (`Heliocentric') Corrections \leaderfill 31 \qquad}
\line{\qquad\qquad\qquad Galactic and Ecliptic Coordinates \leaderfill 31 \qquad}
\line{\qquad\qquad\qquad Bugs and other problems \leaderfill 32 {\it ff.} \qquad}
\line{\qquad\qquad\qquad Programmer's Notes \leaderfill 33 \qquad}
\smallskip
\line{\qquad 2. A Nighttime Astronomical Calendar \leaderfill 35 \qquad}
\line{\qquad\qquad General description \leaderfill 35 \qquad}
\line{\qquad\qquad Times in the Calendar Program \leaderfill 36 \qquad}
\line{\qquad\qquad Running the Calendar \leaderfill 37 \qquad}
\line{\qquad\qquad More on TeX Output \leaderfill 38 \qquad}
\line{\qquad\qquad Sample Output \leaderfill 39 \qquad}
\smallskip
\line{\qquad 3. Cautions Applying to Both Programs; Miscellany. \leaderfill 39 {\it ff}\qquad}
\par\vfill\eject


\centerline{\bf 0. INTRODUCTION -- WHY BOTHER?}
\medskip\par
You've just received your time assignment for Kitt Peak and you wonder
whether the moon will interfere with your objects during those nights,
which {\it weren't} the one's you asked for.
Or, you're sitting at the telescope at 1 AM, wondering if you can squeeze 
in a 1-hour exposure before twilight at acceptable
airmass on an object that's just rising now.  Maybe you want to
set your spectrograph slit to neutralize atmospheric dispersion.
Maybe you want to precess a couple of objects' coordinates, or see
what their galactic latitude is.  Perhaps you want to spot-check the
canned heliocentric corrections which IRAF has applied to all your
data.
Perhaps you just want to know how high the sun will be above the horizon
at 4 PM in October, so you can see if it's safe to take a bike ride.
Any competent astronomer armed with some reference materials and
a calculator
can answer these questions, but it takes time.  Over the years I've
done this sort of thing many times; I finally decided to encapsulate
some utility routines of this sort into a couple of convenient,
easy-to-use, portable packages.
\par
This document describes two programs.  The more
powerful and interesting one is an interactive 
astronomical circumstances calculator. 
The other prints a
1-year nighttime {\it calendar\/} of phenomena for a single site; this
will generally be run in background, to produce a table to be
hung on the observatory wall or put in a notebook.
\par
I wrote both these as standalone C-language programs.  To maximize
portability and ease of use I tried to make the user interface 
``as simple as possible, but no simpler'' (to paraphrase Einstein).  There
are no graphics, no mouse-driven menus, or anything like that.
You type stuff and the computer types stuff back.  
The commands are as terse as possible -- single letters -- so even
hunt-and-peck typists should be able to use the programs efficiently.
Throughout this document I've indicated things that you or the
computer type with {\tt this typeface}.

In some sense these programs and their documentation are 
a `publication' for me, though not in a refereed journal.  Accordingly,
I'd like these to be disseminated as widely as possible in the 
community of professional astronomers.  Please feel free to pass them
along.  If you would like your own copy, it can be obtained via
anonymous ftp from {\tt iraf.noao.edu}; it's in the {\tt contrib} directory
under the name {\tt skycalc}.  You'll need access to a UNIX machine to unpack the
{\tt tar} file, but the code itself should run on almost anything with
a C compiler. 
\par
\bigskip
\centerline{\bf 1. THE INTERACTIVE ALMANAC}

\begsubsect{Overview}

This is designed to provide quick and easy access to
astronomical quantities of interest to an observer at the telescope,
and to ease the planning of (especially nighttime) observations.

This may seem unnecessary, since so many powerful `desktop planetarium' 
programs are available.
While these are very impressive, and often very useful, they
don't always provide the information needed by professional astronomers
in the most useful form.  Furthermore they are generally wedded to a 
particular architecture and operating system, generally PCs or
Apple machines; Elwood Downey's magnificent {\it xephem} program 
is an exception, but it's aimed somewhat differently than the present
program.  

I wrote this program to serve my own needs, which
are broadly typical of the professional observing community.
Many amateurs may find it useful as well.
It is written in garden-variety
C and has a dead-simple user interface (you type, then it types -- no
mouse, no graphics, nothing difficult to move from machine to machine).
Commands are terse.
The code is designed to run on workstation-class machines ubiquitous
in professional circles; modifications are necessary to fit it onto
PC-class machines, unless they're running Linux, on which it runs
perfectly.  There is a provision to write calculated results
directly to a log file without having to use operating-system
output redirection.

The program is designed as follows.  You specify information about 
your site, the coordinates of your object, the date and time, or
whatever else is relevant, using very simple commands and a flexible,
obvious format.  Then you give a command to do calculations
and put out results; some of these commands prompt for further
needed information.  The output commands are
explained in more detail later, but the following summary gives some idea
of the program's aims:

\item{\tt h } The `hourly airmass' command prints a table of the 
airmass, hour angle, and other information for each hour during the
night.  Users tell me that they use this more frequently than any
of the other options.  It uses the date, the site, and the object coordinates. 

\item{\tt a } The `nightly almanac' command tabulates information
about a single night, including times of sunset, sunrise, twilight, moonrise
or -set, the moon phase, moon coordinates, the moon phase at midnight,
sidereal times at midnight and twilight, and other such.  It uses
only the site information and the date.

\item{\tt = } This prints the `instantaneous circumstances' for your
observation; it uses practically all the input information (site, RA and dec,
date and time) and tells you the airmass, precessed coordinates, the state
of the moon and its modeled sky-brightness contribution, twilight,
heliocentric (barycentric, actually) corrections, the parallactic angle,
the julian date, and various other stuff.  There are little niceties
such as a check as to whether a major planet might be near the line of
sight, and whether a solar or lunar eclipse is in progress.

\item{\tt o } The `seasonal observability' command is designed to help
you accurately assess the `range of acceptable dates' for observing a 
given object.  It tabulates how many hours an object will be observable
at night at less than 3, 2, and 1.5 airmasses from your site; the 
tabulation is for each new and full moon between two specified dates.
Thus this serves as a lunar phase calendar as well.

\item{\tt m } This `major planets' command types out rough positions of 
the major planets for your site, date, and time, including their
hour angles and airmasses.  

In addition, there are several special-purpose calculators invoked by
two-character commands 
(the two-character commands are called `extra goodies').  These list
times of events for a variable star, precess batches of coordinates,
convert julian to calendar dates, give the offset from ephemeris
time, and such.

Specifying input parameters one-by-one can be tedious, so there are a
number of ways to make this more convenient.  Sites (which include
most of the world's major observatories) are presented
on a menu.  One can read the
system clock with {\tt T} and set the date and time to `right now'
(plus a settable offset, in case you're interested in, say,
half an hour from now).  This is especially useful at the telescope.
There is also provision under extra goodies
for reading in a list
of objects (in a simple format), presenting this list sorted in
various ways, and selecting object coordinates from it.  
Finally, one can automatically 
set the coordinates to the zenith for the specified site, date,
and time.

There's flexibility as to how you specify dates and times, which
comes at a cost in consistency and simplicity.  By toggling
software switches you can specify times either in UT or local zone time
(which can optionally include daylight savings); also, you can
apply a date convention by which the evening date applies all night,
for nighttime continuity.
 
Later in this document I give a lengthy and detailed description 
of the level of
accuracy expected for all the calculations.  My philosophy has been
to compute everything as accurately as I could, consistent with the
requirement that the program be self-contained and portable.  For example,
the precession and sidereal time calculations are very
accurate because they are reasonably compact, but the planetary positions 
are not definitive because that would require the 
inclusion of rather extensive data tables.  Which leads to $\ldots$

\par
{\it	
** A FEW CAUTIONS: I've made this code as accurate and
as generally useful as I could, but
before using it for purposes where extreme accuracy is critical, or for
locations at extreme geographical positions, it's a good
idea to read up on the algorithms and their limitations.  And it's
always the user's responsibility to be sure the answers are sensible.
Be especially careful when the code is recompiled on your local machine;
experience shows that compilers can generate different answers from
the same code.  The examples below can be used to check that
your local compiler gives good answers.
With these caveats, we proceed to.... **
}
\bigskip
\centerline{\bf 1.1 -- BASIC USE OF THE PROGRAM.}
\par
Before starting, note that you should ideally be
able to get going with the program in about 10 minutes
{\it without} referring to this document, using the fast guided
tour option and other help text.  
(I've established this empirically with volunteer astronomer
subjects).  Nonetheless, going through the program with this
document in hand should give a more thorough understanding of the
program and draw attention to its capabilities.
This document should be a useful reference, especially for
questions of accuracy, generality, and such.
I assume that the user is familiar with
the concepts of celestial coordinates, sidereal time, and 
so on; the program has no provision for teaching complete novices!

Note that the program doesn't prompt you under most circumstances, 
which can be disconcerting; but typing a 
few empty carriage returns in a row will
usually point you towards help.  

\par

\begsubsect{Starting Up}

Begin by running the program.  How you do this is dependent
on your operating system.  On a UNIX system, you just
type the name of the program, which is likely to be 
{\tt skycalc}, assuming it is in your current directory or
has been installed in your path.
\par
The program first asks you to select an observing 
site; a little menu comes up which gives single-character
codes for a number of major (and minor) observatories.
Simply type the single character for the site you want, followed
by a carriage return.  (Throughout the program, nothing happens
until you type a carriage return).  The examples computed below
are for Kitt Peak ({\tt k}), but you can give whatever you like.
Note that the input
is `case sensitive', so {\tt K} is not the same as {\tt k}!.  
Here's what the menu should look like, though you'll 
undoubtedly have more menu options:
\par

{\tt\obeyspaces
\line{\hfil }
\line{\qquad Astronomical calculator program, by John Thorstensen. \hfil}
\line{\hfil }
\line{\qquad   *SELECT SITE* - Enter single-character code: \hfil}
\line{\qquad       n .. NEW SITE, prompts for all parameters. \hfil}
\line{\qquad       W .. Write site parameters to a file. \hfil}
\line{\qquad       R .. Read new site parameters from a file. \hfil}
\line{\qquad       x .. exit without change (current: Kitt Peak) \hfil}
\line{\qquad       k .. Kitt Peak [MDM Obs.]\hfil}
}	

\par

... (and so on ... several different sites are available) ...
\medskip\par
{\tt\obeyspaces
\line{\qquad       l .. Lick Observatory \hfil}
\line{\qquad  Your answer --> \hfil}
}

\par

[If the desired site is not on the menu, type the letter
for a new site ({\tt n} in this example) and you'll be prompted for the
characteristics of the site.  The prompts should be
self-explanatory.  Note that the longitude and time zone are
{\sl positive westward}, unlike the Almanac convention.  Also, you
must specify the longitude in {\sl hours} minutes and seconds, 
and the latitude in {\sl degrees} minutes and seconds.
Once you've entered new site parameters, you can check them with the
{\tt l} (parameter list) command, then save them in a
little ASCII file by typing {\tt s} to reinvoke the site parameter command
and then typing {\tt W} to write a file.
The {\tt R} option in the site menu prompts for a filename.]
\par

The program next attempts to read
your computer's internal clock to establish the time and date.
Thus the program should wake up with the time and date
set to `right now'
\footnote{*}{This works on all systems I've tested so far,
and the c library functions I use for this are supposedly standardized.
In case it doesn't work I've built a switch into the 
source code to disable all the functions which rely on the system clock;
if the system
clock is disabled, it will tell you at this point and set the time
and date to a default value of 2000 Jan 1 at midnight.}
.   This may not be what you want, but among all possible times
it's perhaps the most likely choice, so it's the default.
The machine will tell you what it has set for time and date.

After this, the program sets the RA and dec used for computations
to the zenith for the specified site at the date and time which have
just been set.  This again may not be what you want, but if
you're at the telescope it at least assures that the coordinates
are above the horizon, and you can easily change this later.  The default
coordinate epoch is 2000.

Finally, the program suggests that new or rusty users take the 
`fast tour' sequence by typing {\tt f}.  
This should take about 10 minutes.  The
rest of this discussion follows this tutorial introduction. 

\begsubsect{Specifying a date ({\tt y}) and getting an almanac ({\tt a})}

The guided tour first suggests you specify an evening date 
(as year month day) and
get the almanac for that date by typing
\par
{\tt \qquad y 2002 4 4  a}
\par
(followed, as always, by a carriage return).

The {\it command-line syntax} of this program, such as it is,
is nicely exemplified here.  The {\tt y} command means
`set the date to the following date, expressed y m d'.  The
reason for the somewhat non-obvious choice of  {\tt y} to specify
the date is that {\tt d} is used for declination; since the date 
format starts with the year, this is at least a little bit mnemonic. 
The {\tt a} command means `print the almanac information for the
presently specified (evening) date'.  Note that the commands are
case-sensitive, so {\tt Y} will not work in place of {\tt y}.
The program seldom cares where carriage returns are placed,
but doesn't do anything until a carriage return is typed.  
This command produces the following:
\parskip 0pt plus 0.5pt minus 0pt
\verbatim$

*** Almanac for the currently specified date ***:
Almanac for Kitt Peak [MDM Obs.]:
long.  +7 26 28 (h.m.s) W, lat. +31 57.2 (d.m), elev.  1925 m
Mountain Standard Time (  7 hrs W) in use all year.

For the night of: Thu, 2002 Apr 4 ---> Fri, 2002 Apr 5
Local midnight = 2002 Apr 5,  7 hr UT, or JD 2452369.792
Local Mean Sidereal Time at midnight =  12 27 11.4

Sunset (  700 m horizon):   18 52 MST; Sunrise:   6 06 MST
Evening twilight:  20 12 MST;  LMST at evening twilight:   8 38
Morning twilight:   4 47 MST;  LMST at morning twilight:  17 15
12-degr twilight: 19 42 MST -->  5 16 MST; night center:   0 29 MST

Moonrise:   2 19 MST
Moon at civil midnight: illuminated fraction 0.430
0.6 days after last quarter, RA and dec:  19 43 32, -24 32.7

The sun is down for 11.2 hr;  8.6 hr from eve->morn 18 deg twilight.
 6.1 dark hours after end of twilight and before moonrise.

$
\medskip
\parskip 2.5ex plus 0.5ex minus 0.5ex

Note that the date you have
given is interpreted as the the {\it local\/} date for {\it evening\/}; 
the rest is largely self-explanatory, but a few remarks are in order.  
The times of moonrise or moonset are reported only if they occur
when the sun is down (or close to it).
The phase of the moon is printed, together with its illuminated fraction
and celestial coordinates; these are computed 
for local midnight, whether or not the moon is
actually up at the time.  If the observatory elevation is non-zero,
an approximate correction is applied to the times of moonrise and
moonset; this is discussed more fully in the section on algorithms
and accuracy.	
\par

\begsubsect{Specifying ra and dec ({\tt r} and {\tt d}); hourly airmass ({\tt h})}

Now let's get a little more specific and consider observing a
particular object.  The guided tour now suggests you specify an
RA and dec and make an hourly airmass table by typing
\par
{\tt \qquad r 15 38 29.2 d -0 01 02  h}
\par
Note here that the default epoch for input coordinates is 2000; 
you can change the input epoch by typing, say, {\tt e 1950}.

As the example shows, times, 
right ascensions, and declinations are generally entered
as {\it triplets of numbers} separated by `whitespace'
characters (blanks, tabs, or newlines).   
Colons will also work
as delimiters, so you can cut-and-paste from other output.  
When you use blanks to delimit the fields, 
the leading parts (hours, degrees, minutes) of right ascensions and 
declinations can have a 
fractional part; for instance, RA ${\rm 19^h 15^m 00^s}$ could be entered as
\par
{\tt \qquad r 19.25 }
\par
and you don't have to enter the trailing zeros, 
provided your next character is a valid command (such as {\tt =}), and 
there is at least one blank following your number.  
The fractional hour and minute feature won't work with 
colon-separated input, but you can leave off the seconds; 
{\tt 18:23} will be properly interpreted, but {\tt 18:23.5} will
be read {\it incorrectly} as 18~23~05.   
To enter a negative 
declination, just make the first number negative, as in 
\par
{\tt \qquad d -0 18 30}
\par
($-0$ is correctly handled to give a negative declination, but there
cannot be any space between the minus sign and the number following.)

The {\tt h} command, which generates the hourly airmass table, first 
prompts you for the {\tt Name of object}.  The reason for this is 
that you may wish to redirect output from the program (discussed 
later) to make a hard copy; the name then serves to label the output.
The name serves only as a label, so you can give a random character
if you want.  Here's what the output looks like:
\parskip 0pt plus 0.5pt minus 0pt
\verbatim$

*** Hourly airmass for Flapdoodle's Variable Nebula ***

Epoch 2000.00: RA  15 38 29.2, dec  -0 01 02
Epoch 2002.26: RA  15 38 36.1, dec  -0 01 28

At midnight: UT date 2002 Apr 5, Moon 0.43 illum,  64 degr from obj

  Local      UT      LMST      HA     secz   par.angl. SunAlt MoonAlt

  19 00     2 00     7 26    -8 12   (down)   -53.4     -3.3    ...
  20 00     3 00     8 27    -7 12   (down)   -56.7    -15.7    ...
  21 00     4 00     9 27    -6 12   (down)   -58.0     ...     ...
  22 00     5 00    10 27    -5 12    5.644   -57.5     ...     ...
  23 00     6 00    11 27    -4 12    2.588   -55.0     ...     ...
   0 00     7 00    12 27    -3 11    1.757   -49.9     ...     ...
   1 00     8 00    13 27    -2 11    1.403   -41.0     ...     ...
   2 00     9 00    14 28    -1 11    1.238   -26.1     ...     ...
   3 00    10 00    15 28    -0 11    1.180    -4.4     ...      5.7
   4 00    11 00    16 28     0 49    1.207    18.9     ...     15.4
   5 00    12 00    17 28     1 49    1.327    36.4    -15.3    23.6
   6 00    13 00    18 28     2 50    1.597    47.2     -2.9    29.8

$
\medskip
\parskip 2.5ex plus 0.5ex minus 0.5ex

Each line shows the local time, the UT, the local mean sidereal time, 
and the object's hour angle; the next
quantity $\sec z$, the secant of the zenith angle, is essentially
the same thing as the airmass.   The notation {\tt (down)} in
this column means the object is below the horizon; {\tt v.low}
will occasionally appear in this column, meaning that the
object is so near the horizon that $\sec z$ will overflow the
space provided for it.   Note that the last two columns
give the altitude of the sun and moon; the sun is printed if 
it is higher than $-18^{\circ}$, and the moon if it is higher
then $-2^{\circ}$.  Otherwise ellipses (${\ldots}$) are 
printed in those spaces.  Also notice that the line giving
the UT date at (local civil) 
midnight also gives the moon's illuminated fraction and its angular
distance from the object.  These are computed for local midnight, whether
or not the moon is actually up then. 

\begsubsect{A Word about Hard Copy and Log Files}

Users tell me that the {\tt h} command is the most commonly
used output feature, and that it's especially nice to have hard copies
of the tables for your main targets.
This therefore seems like a good place to mention
two ways to direct the program's output to a file:
one can either redirect output
using general operating-system features (like invoking the {\tt script}
command on UNIX before running the program) , or, if your system supports it,
you can use a {\it log file} feature.  The log file should be the more
convenient of the two possibilities.  [{\bf However}, 
it has proven troublesome
on some systems (such as, old Sun Sparcstations
running Sun's vanilla-flavor non-ASCI C compiler).  Because of these
troubles, a version of the code is distributed which does not
include the log file option.  There may be other circumstances in which
your system manager decides to disable the log files.  Typing 
{\tt x?} will tell if the option is available -- if you see an
{\tt xL} option, you have it.]
  
The log file feature (if enabled)
is invoked from the `extra goodies' menu by
typing {\tt xL}.  The big {\tt L} in this case is meant as a mnemonic
that something big is happening (generally, I reserve upper-case
for commands which change more than one thing at a time).  You're
prompted for the name of a log file.  {\it When the log file is
open, almost all the program's
output (except for prompts) is also written to the log file.}
If you print out the online documentation -- such as the menu and fast tour  --
this goes to the log file as well.  
The log file has minor format differences from the terminal output to 
make it easier to read.  If you type {\tt xL} again while the log file
is open, it closes the log file, so {\tt xL} acts as a toggle.  The
log file is opened with `append' permission, so if it exists already
new output is written to the end. 

More general output-redirection using the operating system is not
as simple (unless your system supports something like the UNIX
{\tt script} command), and 
details are system-dependent.  You'll
generally have to prepare a list of commands to feed into the
program.  Here's a sample
of what the command input file might look like in a typical case
for the {\tt h} command; the comments are for human legibility
and are not to be included.
\parskip 0pt plus 0.5pt minus 0pt
\verbatim$

(input)               (comment)

k                     (selects Kitt Peak, assuming it's in site menu).
y 1990 10 20            (date)
e 2000                (select epoch of input coordinates)
r 19 19 19            (coordinates .. ra and dec).
d 2 2 2	
h                     (hourly circumstances command)
Wholeflaffer 9	      (name of object .. the output will follow.)
r 20 20 20            (for next object)
d 12 12 12	
h
Flapdoodle's Variable Nebula
$
\medskip
\par
and so on... {\tt r, d, h,} 
followed by the name, until..)
\medskip
\verbatim$
Q              (exit program).	
$
\parskip 2.5ex plus 0.5ex minus 0.5ex

(Note: A previous version of the {\tt h} command prompted for the number of
hours to print; this is now computed automatically, so old scripts 
designed for the previous version will have to be revised slightly.)

\begsubsect{Instantaneous Circumstances; the {\tt =} Command}

The next action suggested in the guided tour is to specify a 
time of day, and then display the instantaneous circumstances
by typing, for instance,
\par
{\tt \qquad t 5 10 0 =}
\par
By default, the time you enter is taken to be the local time,
but this can be changed to UT with the {\tt g} option (below).
Also by default, a morning time (such as this one) is interpreted
as referring to the morning of the date {\it after} the specified
date; this way, morning and evening times refer to the {\it same
night}.  This can also be changed, using the {\tt n} (`night-date')
option.  These are all explained later.
\par
The {\tt =} causes the instantaneous circumstances to be displayed;
for the present parameters, the output is:
\parskip 0pt plus 0.5pt minus 0pt
\verbatim$

W Long (hms): +7 26 28.0, lat (dms): +31 57 12, std time zone   7 hr W

Local Date and time: Fri, 2002 Apr  5, time   5 10 00.0  MST
   UT Date and time: Fri, 2002 Apr  5, time  12 10 00.0
Julian date: 2452370.006944   LMST:  17 38 02.4

Std epoch--> RA: 15 38 29.2, dec:  -0 01 02, ep 2000.00
Current  --> RA: 15 38 36.1, dec:  -0 01 28, ep 2002.26
HA:  +1 59 26; airmass =    1.358
altitude  47.36, azimuth 227.31, parallactic angle 38.6  [-141.4]

In twilight, sun alt -13.3, az  74.0 ; Sun at  0 57 15.1,  +6 06 56
Clear zenith twilight (blue) approx  2.2  mag over dark night sky.
Moon: 19 53 48, -24 38.4, alt  24.8, az 146.0; 0.412 illum.
0.9 days after last quarter.  Object is  66.3 degr. from moon.
Lunar part of sky bright. =  21.7 V mag/sq.arcsec (estimated).
Barycentric corrections: add  380.7 sec, 17.03 km/sec to observed values.
Barycentric Julian date = 2452370.011350

$
\medskip
\parskip 2.5ex plus 0.5ex minus 0.5ex
Notice all that has been computed!
\par
After a brief summary of the site information, the next block
establishes the time in various systems.
The date and time are given in both local and UT.  If daylight
savings time is selected, a recipe which should be appropriate to the
site is used to select whether the local time
is reported in DST or Standard.  Note that the label on the local
time will have either an `S' or a `D' as the second character,
to indicate standard or daylight.  LMST stands for the
{\it local mean sidereal time}, which is essentially the
local sidereal time.  It disagrees with the true
hour angle of the equinox by the `equation of the equinoxes', caused 
by nutation; this amounts to a couple of seconds at most.
\par
The next block refers to the object.
The object's coordinates are reported both for
the `standard epoch', which is set with the {\tt e} option, and for
the mean equinox of date.  The program `wakes up' assuming that input 
coordinates are for equinox 2000.
Proper motions may be included (see below).
\par
If the coordinates given are more than about 5 degrees above the horizon,
the airmass is printed; at lower elevations the quantity secant of the 
zenith distance is
printed, since the polynomial expression used to compute airmass from
secant $z$ becomes inaccurate.
If the object is at a particularly large airmass, or below
the horizon, a comment is printed.  If the object's secant $z$ 
is very large, it is not printed to avoid overflowing the space provided.
The {\it altitude\/} is 90 degrees minus the zenith distance, uncorrected
for refraction;
the {\it azimuth\/} is (as usual) measured from the north 
through the east. The {\it parallactic angle\/}
is the position angle (measured N through E at the object)
of the arc that connects the object to the zenith, or loosely
speaking, the position angle of `straight up'.  This is
useful for setting a spectrograph slit to catch the dispersed
light. (See Filippenko, 1982, PASP, 94, 715 for a discussion).
The parallactic angle may change sign at the meridian, but actually
varies smoothly.  Because some applications (e.g., the placement of
a spectrograph slit) are indifferent to a
180-degree shift in the parallactic angle, the angle $\pm 180$ degrees
is reported in square brackets.
\par
If the moon could be interfering (higher than 2 degrees below the
uncorrected geometrical horizon), its phase, 
fraction illuminated, approximate RA, dec, altitude (not corrected
for refraction), and azimuth are 
reported.  Also, the angle subtended at the observer by the object 
and the moon is reported.  If the moon is more than two degrees below
the horizon, it is reported to be `down' and only its phase is
printed.  If both the moon and the object are above the horizon, 
{\it and} the sun is more than 9 degrees below the horizon, an
estimated value of the moon's contribution to the night-sky brightness
is given; this is obviously only approximate, and only holds under
ideal conditions.  For comparison, a dark site has about 21.5
$V$ magnitude per square arcsec, but this varies considerably.
\par
If the sun is at a geometric altitude $> -18^{\circ}$ but below the
horizon, twilight is reported.  The zenith distance at which the
sun's upper limb reaches the horizon is taken to be $90.83^{\circ}$
if the observatory elevation is zero; the extra $0.83^{\circ}$
account approximately for the refraction and average angular 
semidiameter of the sun.  A further correction appropriate
to a sea-level horizon is added if the site's elevation is nonzero.
In twilight, an estimate of the brightness of twilight at the 
zenith is reported; these numbers appear to match rather well
the behavior of twilight in blue light.  In visual or red, the 
enhancement due to twilight may be rather fainter.
If the sun's upper limb is above the horizon, 
it is reported to be `up'.  In twilight or daylight 
the RA, dec, altitude, and azimuth of the sun are given.
This altitude is not corrected for refraction.  If the sun's zenith distance
is greater than $108^{\circ}$ (or more than roughly
eighteen degrees below the horizon), it is reported to be `down'.
\par
Another feature checks to see if the position you've specified
is within $3^{\circ}$ of the computed (low-precision) position 
of any major planet.  If it is, the program warns you.  This way
you won't try to set on your 98th magnitude object only to find that Jupiter
is 5 arcminutes away!
Yet another feature reports if an eclipse of the
sun or moon in progress.  The accuracy expected of these predictions
is discussed later.

\begsubsect{\it The {\tt o} command - Observability through a season}

The next suggestion in the guided tour is to explore the
observability of your object through an observing season,
by typing {\tt o}.  
The output is designed for use {\it before} you apply for telescope time --
it supplies you with accurate information to allow you to decide the
`range of acceptable dates' by printing a summary of the observability
of your object (as specified by the RA, dec, and epoch)

You are first prompted for the range of dates to cover,
(in standard y m d form).  A  6-month span fits
in a standard 24-line screen, if that it important.
It next asks for the altitude of the sun to be used for
twilight; $-18^{\circ}$ is standard astronomical twilight,
but this is very dark, so you may wish to relax this
condition some if you can live with a little sky light.     
Finally, it prompts for an object name (as in {\tt h} above),
simply to use as a label for any hard copy you might make.
The object we've been working with (at about 15 hours) would
be in the first semester of the year.  Typing  
{\tt 2002 1 30} and {\tt 2002 7 30}, and standard twilight
(plus a name to label the output) gives the following output:
\parskip 0pt plus 0.5pt minus 0pt
\verbatim$

          *** Seasonal Observability of Flapdoodle's Variable Nebula ***

     RA & dec:  15 38 29.2,  -0 01 02, epoch 2000.0
Site long&lat:  +7 26 28.0 (h.m.s) West, +31 57 12 North.

Shown: local eve. date, moon phase, hr ang and sec.z at (1) eve. twilight,
(2) natural center of night, and (3) morning twilight; then comes number of
nighttime hours during which object is at sec.z less than 3, 2, and 1.5.
Night (and twilight) is defined by sun altitude < -18.0 degrees.

 Date (eve) moon      eve            cent           morn     night hrs@sec.z:
                    HA  sec.z      HA  sec.z      HA  sec.z    <3   <2   <1.5
2002 Jan 28   F  +11 49  down    -6 52  down    -1 33   1.3   2.9   2.0   1.0
2002 Feb 11   N  -11 05  down    -5 56  62.4    -0 46   1.2   3.7   2.8   1.8
2002 Feb 26   F   -9 55  down    -4 58   4.4    -0 02   1.2   4.4   3.6   2.5
2002 Mar 13   N   -8 44  down    -4 02   2.4    +0 40   1.2   5.1   4.2   3.2
2002 Mar 27   F   -7 39  down    -3 11   1.8    +1 16   1.2   5.7   4.8   3.8
2002 Apr 11   N   -6 27  down    -2 16   1.4    +1 54   1.3   6.3   5.5   4.4
2002 Apr 26   F   -5 14   5.9    -1 20   1.3    +2 32   1.5   7.0   6.1   5.1
2002 May 11   N   -4 00   2.4    -0 23   1.2    +3 14   1.8   7.2   6.8   5.1
2002 May 25   F   -2 51   1.6    +0 33   1.2    +3 57   2.3   6.8   6.4   5.1
2002 Jun 10   N   -1 36   1.3    +1 39   1.3    +4 54   4.1   6.0   5.2   4.1
2002 Jun 24   F   -0 36   1.2    +2 37   1.5    +5 50  28.3   5.0   4.2   3.1
2002 Jul 9    N   +0 20   1.2    +3 39   2.0    +6 58  down   4.1   3.2   2.2
2002 Jul 23   F   +1 06   1.2    +4 35   3.3    +8 05  down   3.3   2.5   1.4
2002 Aug 7    N   +1 49   1.3    +5 34  10.3    +9 18  down   2.6   1.8   0.7

$
\medskip
\parskip 2.5ex plus 0.5ex minus 0.5ex

Because observing time requests are so intimately tied to 
lunar phase, the dates selected are those of full and new moon; they are
selected to be those (local evening) dates on which new or
full moon fall within 12 hours of the center of the night.
The tabulation starts with the lunation before your specified starting
date.  At each date, the object's hour angle and airmass (actually $\sec z$)
is given (1) at evening twilight, (2) at the natural center
of the night, and (3) at morning twilight.  The `natural center'
is time of the sun's lower culmination (when its hour angle is 
12 hours); in general it differs from local clock 
midnight because of location
in the time zone, daylight savings time, and the equation of 
time.   Finally, the last three columns give the number of 
hours during the night (that is, past twilight) for which 
the object is at airmasses less than 3, less than 2, 
and less than 1.5. 
These limits are arbitrary, but representative of poor, marginal, 
and good observability. Circumpolar objects can be observable 
both at the beginning and the end of a long winter night; the code 
appears to tally the observable hours properly.  

In high latitudes, twilight does not occur in midsummer; in this
case, {\tt twi.all.night!} appears in the columns
for position at evening and morning twilight.  At 
extremely high latitudes, the sun can remain below the specified twilight 
altitude all day, and these columns then contain 
information for the times at which the sun is $\pm 12$ hours from 
its lower culmination.
                               
\begsubsect{Looking at current parameters with {\tt l}}

At this point we've set a fair number of parameters.
While many of the the current parameters are printed in the
output from {\tt =}, others are implicit, and the display is
crowded, so they're hard to keep track of.
Thus the {\tt l} (`look') command simply prints out a nicely-formatted list
of input parameters.  Its output is
\parskip 0pt plus 0.5pt minus 0pt
\verbatim$

Current INPUT parameter values:

      DATE: 2002 Apr 4
      TIME:  5 10 00.0

NIGHT_DATE:  ON    -- date applies all evening & next morning.
  UT_INPUT:  OFF   -- input times taken to be local.
   USE_DST:   0    -- Standard time in use all year.
AUTO UPDATE: OFF  -- system clock not automatically read on output.

            RA:  15 38 29.20
           DEC:  -0 01 02.0
   INPUT EPOCH:    2000.00
PROPER MOTIONS:  OFF

SITE: Kitt Peak [MDM Obs.]
      E.longit. = -111 37.0, latit. = +31 57.2 (degrees)
      Standard zone =   7 hrs  West
      Elevation above horizon =  700 m, True elevation = 1925 m

$
\medskip
\parskip 2.5ex plus 0.5ex minus 0.5ex

This is particularly useful for keeping track of the {\tt g} and 
{\tt n} commands, which cause the interpretations of time and
date to {\it toggle} between different cases.  Because the effect
of each of these commands depends on the status when they
are executed, 
it's helpful to be able to 
look at their state without doing anything else.  Also, the site 
latitude and longitude are converted here to a format which
exactly matches the numbers in the {\it Astronomical
Almanac} observatory list, to make it easy to check them.

\par\vfill\eject
\centerline{\bf 1.2 -- MORE COMMANDS.}
\bigskip

\begsubsect{Quitting the program -- {\tt Q}}

This stops the program gracefully.  
The {\tt Q} must be upper-case -- this 
should avoid accidents well enough.

\begsubsect{Printing a menu -- {\tt ?}}

This causes the following menu to print out.
\parskip 0pt plus 0.5pt minus 0pt
\verbatim$
Circumstance calculator, type '=' for output.
Commands are SINGLE (lower-case!) CHARACTERS as follows:
 ? .. prints this menu; other information options are:
i,f . 'i' prints brief Instructions and examples, 'f' fast tour
 w .. prints info on internal Workings, accuracy & LEGALITIES
TO SET PARAMETERS & OPTIONS, use these (follow the formats!):
 r .. enter object RA, in hr min sec,  example: r 3 12 12.43
 d .. enter object Dec in deg min sec, example: d -0 18 0
 y .. enter date, starting with Year   example: y 1994 10 12
t,T: t = enter time, e.g.: t 22 18 02 [see 'g' and 'n']; T = right now+
 n .. *toggles* whether date is used as 'evening' (default) or literal
 g .. *toggles* whether time is used as Greenwich or local
 e .. enter Epoch used to interpret input coords (default = 1950)
 p .. enter object Proper motions (complicated, follow prompts).
 s .. change Site (again, follow prompts).
 l .. Look at current parameter values (no computation).
TO CALCULATE AND SEE RESULTS, use these commands: 
 = .. type out circumstances for specified instant of time, ra,dec
 a .. type out night's Almanac for specified (evening) date
 h .. type out Hourly airmass table for specified date, ra, dec
 o .. tabulate Observability at 2-week intervals (at full&new moon)
 m .. Major planets -- print 0.1 deg positions for specified instant
 x .. eXtra goodies! (galact./eclipt., var star predicts, precess.)
 Q .. QUIT .. STOPS PROGRAM. --->
$
\medskip
\parskip 2.5ex plus 0.5ex minus 0.5ex

I've found in testing that this menu is good for reminding
advanced users of commands, but poor for teaching new users
how to run the program.  Hence it is not mentioned in the
introductory banners, but rather the user is referred to the
fast guided tour, invoked with {\tt f}.

\begsubsect{A word about `extra goodies' -- {\tt x}}

You'll notice an {\tt extra goodies} option on the main
menu.  This option hides some arcane and/or complex commands.
Putting them
here keeps the main menu short enough to fit into a 
24-line display.  If you
type {\tt x?} you'll get a summary which describes these
selections.  The extra goodies command level does not loop,
but drops you immediately back into the main
menu whatever you do, so you will have to prefix any new
extra goodies command with another {\tt x}.

\begsubsect{Setting the time to now -- {\tt T}}

An UPPER CASE T sets {\it both} the time and the date using your
machine's system clock.  (As noted earlier, this option 
may be switched off at compile time if there is some
problem with this.)  There's an interesting twist here:
you are prompted {\tt Set how many minutes into the future?  :}.
Answering {\tt 0} sets the time and date to right now;
any other number sets the time and date into the
future (or past for negative numbers).  This lets you 
quickly answer questions such
as  `Can I get to this object half an hour from now
when I've finished with exposure I'm working
on?'.

The internal actions of this option are modified by the
{\tt g} and {\tt n} options (see below).
The program should do the right thing and set
the date and time to reflect the present (with 
whatever offset you specify) as expressed {\it in the prevailing
time and date convention.}

Note that this option does {\it not} cause the time to be continually
updated; the value of the time set by {\tt T} remains in effect
until you set the time to some other value.  But the next option {\tt xU}
enables this feature.

\begsubsect{Enabling/Disabling Automatic Clock Update -- {\tt xU}}

Typing {\tt xU} one time sets a flag which automatically reads the system
clock whenever you ask for a time-critical calculation; just as with the
{\tt T} command, you are prompted for a time offset when you invoke this
option.  The time offset remains in effect, so you can use a computer
in a different time zone from where you are situated, or keep track of
what will be happening (say) 20 minutes into the future.
Typing {\tt xU} again toggles this option off.  The output options 
which cause the clock to be read are {\tt =}, {\tt m} (major planets), 
{\tt xa} (apparent place), and {\tt xf}
(phase of a repeating phenomenon).  The options {\tt xZ} (set coordinates
to the zenith) and the list-selection options will also read the clock.
The option is toggled off if
you issue a command which indicates that you no longer want it, 
including
setting the time with {\tt t}, setting the date with {\tt y}, setting both
with {\tt xJ}, or reading the system clock explicitly with {\tt T}.
As an example of the effect of this option, when auto updating is toggled
{\it on}, successive
invocations of {\tt =} will come back with slightly different times.

\begsubsect{Changing the Site with the command {\tt s}}

You can change sites by typing the letter {\tt s} and answering the prompts.
When you do this, you will be given a menu of single-character 
site codes from which to choose, just as when you started the program.
Your local version
of the program should be customized to offer the most
common choices for your institution.  To choose a site, just
type the letter (be sure to use the correct lower or upper
case) and hit carriage return.  You can also specify 
a site not on the menu by typing {\tt n} (or the appropriate character
in your customized version).
If you select one of the `canned' sites, all the parameters
(latitude, longitude, time zone info, etc.) will be changed to
their standard values for that site.
\par
If you want a site which is not on the menu, you'll have to give
all its parameters.  Otherwise one would risk of changing 
the parameters piecemeal and having some parameters which are appropriate
to the site and others which are not.  You'll need to know the 
latitude of your site in degrees, minutes, and seconds, 
and the {\it west} longitude in {\it hours}, minutes, and seconds.
(Like Jean Meeus, I dislike the `east longitude' convention, and I
like to use hours for longitude because of the direct connection
with time).
You'll also need the time zone in hours west of Greenwich ({\it e.g.},
Pacific is 8).  You can specify Eastern hemisphere sites by giving
negative numbers for the longitude and time zone.  You'll also be
prompted for the site's elevation above sea level, which 
affects certain quantities slightly, and its
elevation above its horizon, which is used only
to adjust rising and setting times.
\par
The last parameter prompted for is whether {\it daylight
savings time} is to be used in converting between local and UT.  There
are several options given here.  Typing {\tt 0} ignores daylight
time.  Typing {\tt 1} invokes the conventions in use in the
United States; daylight savings starts on the first
Sunday in April and ends on the last Sunday in October from 1986 on,
and from the last Sunday in April before that.  (This ignores
various wrinkles during wars, energy crises, etc.).  Typing 2 gets
you the Spanish (Continental?) convention, with daylight
savings from the last Sunday in March to the last Sunday
in September.  Negative numbers are used for southern sites; 
typing {\tt -1} gets you the Chilean convention
(off daylight savings the second Sunday in March, back
on the 2nd Sunday in October), and {\tt -2} gets the
Australian convention.  Implementing other prescriptions would require 
fairly straightforward modifications
to the source code.  The presently available prescriptions all assume
that the time changes at 2 AM as reckoned in the time preceding the
change, as is standard in the US.

Naturally, you should be sure that you have the correct parameters
for your site.  The {\tt l} command lays out the site parameters
neatly for your inspection, and many are echoed with other output.

Once you're happy with your site parameters, you can save them in 
a file by re-running the {\tt s} command and writing them out with
{\tt W}, after which you can read them with {\tt s} and {\tt R}.
The site files are ASCII; the first line is the site name, the
next the zone name (either of which can have blanks), and the 
last line has the single-character zone abbreviation, the
daylight-savings switch parameter, the longitude in decimal hours,
the latitude in decimal hours, the zone offset, the elevation above
sea level, and the elevation above the horizon terrain.  The
parameters in the file are in the same form as the variables
used in the program (decimal hours, etc.), so it can be used as 
a guide if you wish to hard-code a frequently-used site.

\begsubsect{\it UT time input and `night dates' -- {\tt g} and {\tt n}}

Typing the letters {\tt g} or {\tt n} switches between various options for
the interpretation of input times and dates.  

The {\tt g} command switches between input in UT and in local time.
The program wakes up assuming that dates and times are input
in local time; typing {\tt g} makes the program assume that the
input date and time are in UT; a little message is printed
telling you this.  Typing {\tt g} again switches back to local, and
so on.  Whichever time is assumed for the input is printed
first on output.  Notice that, when you type {\tt g}, the {\it current
time changes}; the input time and date have their same
numerical {\it values}, but are now interpreted differently!  A
message is printed to remind you of this.

Similarly, the program awakens assuming that the date you specify is
to be interpreted as the
{\it evening\/} date for the entire night (this is the `night date'
condition).  For example, if you
print out an almanac for the {\it night\/} of October 20, and then specify a
time after midnight (2 30 00, say) and type {\tt =}, the
circumstances printed are those applying on the {\it morning} of October 21.
The reason for doing this is to maintain some parallelism with the
almanac, which prints the phenomena for a given night.  Typing
{\tt n} once switches this option off, so the current date is interpreted
literally; typing {\tt n} again switches it back on (unless you are
in UT mode), and so on.  This may seem confusing at first, but it
should be less confusing than the alternative.  It is at least
always possible to interpret the output unambiguously; the times
and dates printed there are generated internally directly from the
JD, so they should always be reliable.

The {\tt g} and {\tt n} commands interact.  Going to UT input automatically
turns off the `night date' option, since UT dates should always be
interpreted literally.  You are also prevented from turning on the 
night date condition when UT input is in effect.

  
\par
\begsubsect{\it Coordinate epoch ({\tt e}), Batch precession ({\tt xb}), and apparent place ({\tt xa})}

Typing {\tt e} followed by an epoch
sets the epoch for your {\it input} coordinates.  Setting the
epoch does not in itself cause any coordinates to
be transformed, but rather affects the {\it interpretation}
of the input coordinates when computations are done. 
The {\tt =} and {\tt h} commands (and others) do these
computations, and show both the input coordinates
and the coordinates in the epoch of date.  There's one little hook
here; specifying an input epoch $ < -5000$ (prehistoric) sets the input
epoch exactly to the epoch of the currently specified date.

It's a little awkward to transform lots of coordinates this way, 
because of all the baggage which comes along with it.  Therefore
there is a command in the {\tt eXtra goodies} menu, {\tt xb},
which transforms a batch of coordinates from one epoch to another.
This prompts for input and output epochs, and then for coordinates
in the usual format.  It keeps going until you give it a negative
RA (such as {\tt -1 0 0}).  The batch precession is isolated from
the rest of the program -- it does not affect any parameter values. 

Conventional precession programs transform {\it mean positions} from
one equinox to another, but one sometimes wants the {\it apparent place},
which includes aberration and nutation.  The `extra goodies' command
{\tt xa} computes this for the presently specified coordinates,
proper motion (if any), date, and time.  It also includes an
approximate refraction correction if the object is above the horizon. 
Several stages of the calculation are printed out.
The accuracy of these routines is discussed later.

\begsubsect{\it Proper motions ... the {\tt p} command.}
\par
If you type {\tt p} you will be prompted for annual proper motions of
the object; answer the prompts.  The specification of proper motion is 
complicated because
there are (at least!) two conventions in use for the units of the
proper motion in RA.  One is the annual change of the RA itself, generally
given in seconds of time per year; this is used in the SAO Catalog.
The other is the east-west motion in seconds of arc on the sky,
which is the first times $15 \cos \delta$.  The program will accept
input of either type; if you give seconds of time per annum, you
must follow your value with an {\tt s}, and if you give seconds of
arc you must give an {\tt a}.  The value is converted and passed
internally in the first (time) convention.
Declination proper motions must always be entered as
arcsec per annum.
\par
If either of the proper motions are nonzero, the output of {\tt =} will display

\item{--} the original coordinates in the standard epoch and equinox

\item{--} the coordinates updated for proper motion only (current
{\it epoch}, but standard {\it equinox!\/})

\item{--} the coordinates updated for proper motion {\it and} precession
(current epoch {\it and} current equinox).

as well as the proper motions used.  The reason for doing this is that
with most modern telescopes the coordinate readout can be set to a 
standard equinox, but the actual sky is (of course!) always in the present 
epoch, regardless of what coordinates you apply to it.  So it's
useful at times to display the updated position
without changing the equinox.  Note that the proper motions
are not computed with perfect rigor; the current RA is just the
old RA plus $\mu \Delta t$, and similarly for the dec.  This is
inaccurate very close
to the pole or over very long intervals of time.
\par
\begsubsect{\it The {\tt xc} command -- coordinate conversions.}
\par
Typing {\tt xc} causes the galactic and ecliptic coordinates to be printed.
(As of this writing, there is also a {\tt c} at the main program level
which still does this, but it's not advertised on the main menu,
in order to keep the main menu to 24 lines).
The galactic coordinate algorithm complies strictly with the IAU definition,
which is specified in 1950 coordinates.
If the input coordinates are in a different epoch,
they are precessed internally
to 1950 before being converted to galactic. 
Both conversions work correctly over the entire sky.  
The inverse conversions are not implemented.
\par
\begsubsect{\it The {\tt xj} command -- calculate calendar dates from Julian dates.}
\par
The main program converts calendar to julian dates internally, and 
prints out julian dates with, among others, the {\tt =}
command.  It's sometimes useful to have the inverse,
which converts julian to calendar, 
and the `extra goodies' command {\tt xj} does this calculation.
The command loops until a negative julian date is given.  The routine expects
all the leading digits of the julian date.  If the input is a true
julian date, the output is a UT date.  
The date in the main program is unaffected by this command (see {\tt xJ}
below).
\par
\begsubsect{\it The {\tt xJ} command -- Set to a Julian date.}
\par
Again, this converts Julian to calendar dates, but {\it it also resets the
date and time in the main program} to the appropriate values.
An upper-case {\tt J} is used because two quantities are reset.
The routine takes into account the current input conventions
(toggled by {\tt g} and {\tt n}), so 
the Julian date computed by immediately typing {\tt =} should reproduce
the Julian date you have specified.  [There is one almost unavoidable
{\bf bug}; if daylight savings time is used, then during the double-valued 
hour when daylight savings time switches back to standard time, the program 
interprets the input time by default as standard time; a  
JD during the final hour of daylight time will create
a time which will be interpreted later as standard.  A prominent warning 
is printed in this rare case.] 
Input outside the calendrical limits (1901 -- 2099) is rejected.  Unlike
the calculator-like command {\tt xj}, this one doesn't loop.

\begsubsect{\it {\tt xd} -- Show the value of $\rm TDT - UT$.}

It's occasionally useful to know the difference between UT (based on
the earth's rotation) and TDT (a uniform timescale -- see the Algorithms and
Accuracy section for a more complete explanation).  This computes an 
approximate value of this quantity for the currently specified date.  
The approximations
used should be good to better than a second from 1900 to 2000, and get
increasingly more uncertain in the future (because of the unpredictability
of the earth's rotation).
\par
\begsubsect{\it The {\tt m} command - print a table of the major planets}

As one might expect, this prints a table of the RA, dec, hour angle,
secant $z$, altitude, and azimuth of each of the major planets, as well
as the sun and moon.    
The planetary positions are only modestly precise;
their pedigree and accuracy are explained later. 
The sun and moon calculations are useful for
finding their positions for times when they are below the horizon (or
past twilight for the sun), and would therefore
not be printed with the {\tt =} command.
The output is as follows:
\parskip 0pt plus 0.5pt minus 0pt
\verbatim$

W Long (hms):  7 26 28.0, lat (dms): 31 57 12, std time zone   7 hr W

Local Date and time: Thu, 1995 Mar 23, time   4 50 00.0  MST
   UT Date and time: Thu, 1995 Mar 23, time  11 50 00.0
Julian date: 2449799.993056   LMST:  16 25 31.8

Planetary positions (epoch of date), accuracy about 0.1 deg:

             RA       dec       HA       sec.z     alt   az

Sun    :   0 08.7     0 57    -7 43      -2.77   -21.1   74.8
Moon   :  17 52.1   -20 05    -1 27       1.79    34.0  155.3

Mercury:  23 01.3    -8 46    -6 36      -4.74   -12.2   92.8
Venus  :  21 47.2   -13 45    -5 22      86.46     0.7  106.7
Mars   :   9 06.7    20 11     7 19     -11.61    -4.9  297.5
Jupiter:  16 56.1   -21 49    -0 31       1.71    35.7  171.2
Saturn :  23 15.6    -6 43    -6 50      -4.09   -14.1   89.1
Uranus :  20 08.6   -20 41    -3 43       3.85    15.0  126.8
Neptune:  19 48.3   -20 33    -3 23       3.15    18.5  130.2
Pluto  :  16 04.1    -6 41     0 21       1.29    51.0  188.5 <-(least accurate)
$
\medskip
\parskip 2.5ex plus 0.5ex minus 0.5ex

It's entertaining to note that if your location is on the site
menu, you can get a table of planets for `right now' by starting the
program, giving your site's letter, and typing {\tt m} -- counting
carriage returns, this is four keystrokes.

\begsubsect{Setting to the zenith with {\tt xZ}}

This command simply sets the right ascension and the declination to
those of the zenith for the currently defined
date, time, and site.  It is a capital letter because it changes
two quantities at once.  The coordinates of the zenith are 
precessed from the present epoch (for which they are just the 
sidereal time and the latitude) to the standard input epoch 
defined by the {\tt e} command.

\begsubsect{Object lists -- the {\tt xR, xl, xN, and xS} commands.}

{\it Overview.} These commands were added in 1994 February; they allow one to
read coordinates from a file, list them, and select from them
using various criteria.  These capabilities are quite powerful
but add some complexity, so the commands are hidden among the `extra goodies'. 

At their simplest, these commands allow you to set
the RA and dec to those of a given object, without having to type the
RA and dec into the program.  But they are far more powerful than this;
for one thing, the commands to display and select from the object 
lists also give the {\it hour angle and airmass} for each object 
at the currently defined site, date and time.   This was designed for use 
at the telescope -- after setting the date and time to the present 
(with {\tt T}, perhaps) you can quickly scan a list of objects and 
select one which is well-placed for observation.   The {\tt xS} command 
goes further and presents the list sorted according to various criteria.  

The {\it file format} for objects is straightforward.
Files are expected to contain one object per line; the reading is 
done one line at a time, so an error in one line does not propagate 
to the next.   Here's an example of a correctly
formatted line: 
$$\hbox{\tt v1727\_cyg  21 29 36.2   47 04 08  1950. 18.1  binary xrs - 5hr}$$
The data fields in each line must be separated by
blanks or tabs, but otherwise are free format.  The first field
is a {\it name}, which must be less than 20 characters and cannot
contain any blanks.  It is helpful if you keep the names simple
so you can remember them exactly.
Next comes the RA in hours, minutes and seconds
(three fields), then the dec in degrees, minutes, and seconds (three
more fields).  All these numbers are read as floating point, so
for example {\tt 19 30.5 0} would be equivalent to {\tt 19 30 30}.
If the declination is negative the first character of the degree
field must be a minus sign, and there must be no space between the
minus sign and the remainder of the number.  A {\tt -0} declination 
is handled correctly.  In the eighth mandatory field is the coordinate
epoch.  Finally, the ninth field may contain an {\it optional user-supplied
floating point number}, which might be a magnitude or perhaps a priority
for observation.  Any further entries on the same line are ignored,
so you're free to put notes or other information there (as above).  
This information
will not be read at all; it's only there for your own benefit (say,
for a printed copy of the list).  

If the program does not successfully
read the eight mandatory fields from each line, a complaint is 
printed and the line
is ignored.  If the optional user-supplied floating point number is
not supplied, it is automatically assigned a value of 99.9.  This choice
is arbitrary, but it's 
a rational choice if you're using the field for magnitudes, and it
doesn't crowd any of the later displays.  
 
You are allowed up to 2000 objects in your list.  If you want more, you
can change the defined constant {\tt MAX\_OBJECTS} in the source code
and recompile.  The information stored for each object amounts to 
something like 48 bytes (depending on your machine), so the 2000-object
limit is about 100 k; most users could expand this without running into 
memory limitations.

The {\tt xR} command reads objects from a file; it briefly reviews the
file format and then prompts for
the name of the input file. If you specify the
filename {\tt QUIT} (all upper case), it does not attempt to open
a file.  If you specify a file which the program cannot open, it complains and 
exits back to the main command level.  If the file does open,
but you already have objects
in memory, it asks you whether you would like to append the new 
objects or replace the old ones with the new ones.  The program then
reads the file, complaining if it finds any obvious anomalies (blank lines,
non-number in fields supposed to be numbers, or whatever).  Finally, it
reports how many objects you have, closes the input file, and returns.
If you've filled up to the maximum number of objects, it warns you.

Now the fun begins!  

The simplest thing you can do with the object list
is type out some of the contents; that's done with the {\tt x l} 
(list) command.  You're prompted for the first and last items to 
print out (by number in the list). The program then prints out the
presently defined date and time -- for which the hour angles and
airmasses are computed -- and then simply types
out the information for each object.  The last two columns give the 
hour angle and airmass ($\sec z$) of each object.

The commands {\tt xN} and {\tt xS} are upper-case letters because
they cause more than one quantity to change at once.  {\tt xN}
searches for an object by name (it must be an exact match, including
the upper or lower case of any letters), and
if a match is found the program {\it sets both the RA and dec to that object}.
If the epoch of the object's coordinates in the list is different from the 
currently defined input epoch (the quantity controlled by the {\tt e}
command), the object's coordinates are precessed 
to the input epoch and the coordinates are set to the precessed values
(that is, the program handles this correctly.
It would have been possible instead
to reset the input epoch as well as the RA and dec, but this
would have been even more confusing.)

Like the {\tt xN} command, the {\tt xS} command resets the RA and dec
(with precession if need be, as above),
but now the user gets to select interactively which object to choose.
Ten objects at a time are presented, and
the user selects an object (and sets the RA and dec) by giving its number
on the list.  Typing {\tt m} gives the next 10 objects, while {\tt q}
quits the search without assigning coordinates.  The search continues
until one selects and item or quits, or until the list is exhausted. 

The power -- and fun! -- of this command comes in the order in which the
objects are presented -- the objects are {\tt sorted} according to various 
criteria (hence the choice of letter). 
There are (at present) five different options for the sort:

\item{\tt xS1} sorts the objects in order of increasing distance from the
presently defined RA and dec -- it's  a `find nearest'.  There are 
many ways to use this.  If you set to the
zenith with {\tt xZ}, the objects will be presented in order of
zenith distance.  If you have an object on your list whose coordinates
you remember roughly, but you don't remember exactly what you called it,
you can find it quickly by setting to the rough coordinates and then 
finding the exact match.  You can also use it to find an object close to
the one you're observing to avoid spending too much time slewing or
to match airmasses (but see option {\tt xS3} below).

\item{\tt xS2} sorts the objects in order of the absolute value of the
hour angle.  Therefore it shows you which objects are closest to the
meridian at the presently defined moment of time.

\item{\tt xS3} sorts the objects in order of proximity in airmass to the
present coordinates.  This could be useful to photometrists and
infrared astronomers who may wish to match the airmass between  
program and standard star observation as closely as possible.

\item{\tt xS4} is especially for people chasing objects into the
west -- it queries for a maximum acceptable airmass, then sorts objects
in order of how many minutes it will be before they reach this airmass.
Thus you can see exactly how urgent it is to get to each object.

\item{\tt xS5} sorts the list in order of the optional user-defined
number.  If this is a magnitude, it will be in order of increasing
magnitude; it may be especially useful to put a priority in this
field.  
  
Naturally, only limited information can be given about each item.
After you've selected coordinates, it's advisable to type {\tt =}
to get a complete listing of the observability information.  This is
especially true if the moon is up -- the object you've selected
could be right next to the moon, or even occulted!  

\bigskip

\centerline{\it Some more arcane commands $\ldots$}

\begsubsect{Repeating phenomena -- {\tt xf} and {\tt xv}}

These two commands are useful for periodically repeating phenomena such as
variable-star eclipses.  {\tt xf} prints a table of phases
at regular intervals (such as hourly). {\tt xv} prints 
the times at which a given phase (such as an eclipse) occurs.
Both use the celestial coordinates of the object to adjust
the output to the {\it geocentric} time.  Both commands
give the option of printing information only when the source
is visible at night from the chosen observing site .  
The error bars in the period and time zeropoint
are propagated; {\tt xv} prints error bars in minutes as to
when the phenomenon is predicted to occur, and {\tt xf} prints
the accumulated error in the phase at the tabulated time 
intervals.  Both tasks prompt extensively for input
parameters, but the celestial coordinates and site information
are inherited from the main program.

\begsubsect{{\tt xp} -- Parallax factors.}

This prints out the annual parallax components in RA and 
dec for the currently selected RA, dec, date, and time, in
units of arcsec, for a hypothetical star 1 pc distant; real
stars will have a parallax displacement equal to their 
parallax times these numbers.
The $XYZ$ coordinates of the earth with respect to the
solar system barycenter are also shown.  Finally, the
aberration of light due to the earth's motion is shown.

\begsubsect{{\tt xy} -- Print day of year.}

Computes and prints the day of the year (UT).  The day number
is given as a fraction; the first moment of January 1 UT 
corresponds to day number 1.000. 



\par\vfill\eject
\centerline{\bf 1.3 -- ALGORITHMS, ACCURACY, AND LIMITATIONS.}
\medskip
\begsubsect{\it Calendar and times.}
\par
The time arguments for most of the routines are Julian dates, implemented as
double-precision floating point numbers.  If your machine's double-precision 
mantissa isn't reasonably long, you can run into serious inaccuracy.  
Digital's VAX machines express a JD to a few milliseconds accuracy, but
this should be checked when the code is ported to another architecture.  
Calendar dates and days of the week are derived from a truncation of the 
Julian date, which is the same in each case, so they should always agree.  
\par
As noted earlier, the program makes various transformations to account
for zone time, daylight savings time, and such.  A subtler issue is the
actual timebase which is used for the input time.  The distinctions
between UT, UTC, TAI, TDT are made authoritatively in the 
{\it Astronomical Almanac}, but are widely ignored by astronomers, so
I'll explain them briefly here; this isn't an authoritative 
discussion, but I hope it's essentially correct.  
{\sl Universal Time}, or UT
is Greenwich time based on the true
phase of rotation of the earth.  The earth's rotation gradually
slows with time, and it is 
sufficiently unpredictable that UT can't be 
determined accurately until after the fact.  There are a few
minor variants of UT based on the state of the data reduction in this
determination.
TAI is {\sl International Atomic Time}, which
is the best realizable uniform timescale.  UTC, the famous {\sl coordinated
universal time} broadcast on WWV, is a compromise between these;
it follows UT approximately, but is maintained an 
integer number of seconds away from TAI by the insertion of an
occasional `leap second'.  UTC and UT should be maintained so that
they always agree to within 0.9 sec.  Finally, TDT is {\sl Terrestrial
Dynamical Time}; this is another uniform timescale offset for historical
reasons I don't understand by a constant 32.184 seconds from TAI.  
Strictly speaking, before 1983
the apparent conceptual equivalent of TDT was called Ephemeris Time (ET); I'm
unclear as to the difference between ET and TDT.  On long timescales
UT drifts parabolically away from TDT (or its rough equivalent, ET); 
a perusal of 
pp.~K8 and K9 of the 1995 {\it Almanac} shows that they were equal in 1870
and 1902, and that the difference $\rm \Delta T = TDT - UT$ has now 
reached about a minute.

Because calculations of 
solar system objects should be based on a uniform time scale, the
`argument' of these calculations is generally TDT.  But I ignore
$\rm \Delta T$ in the planetary calculations, because the
planets move rather slowly, and the planetary theory used here
is relatively primitive.  However, TDT is used for the moon calculation
(where it is just significant because the moon moves so quickly) 
and the sun (where at present it changes the answer by about 3 arcsec).
From 1900 to the end of
1997 the values used are based on linear interpolations on 5-year
intervals in the 2000 edition of the {\it Almanac}.  Accuracy appears to be
less than a second when compared to the tabulated annual values.
After 1997, the correction used is a guess, which 
is linearly extrapolated from present-day values.  A parabolic
extrapolation might be better, but the behavior in the past has
often been rather erratic so this seems adequate.

The calendrical routines break down before 1901 and after 2100. Input 
outside those dates causes the program to become uncooperative until you 
set a date inside the allowed range.  While it would be a simple matter 
to extend the calendrical routines, I worry about the wisdom of this 
because I have not tested the accuracy of the celestial calculations far 
outside of the present.  The routine which converts julian date
back to calendar date, which can be accessed directly with the {\tt xj}
command, has a wider range of validity; it agrees with the {\it Astronomical
Almanac} (1995; page K4) from 1600 to 2100 at least.
\par
When printing the phenomena for a given night, the program assumes 
implicitly that zone time at least grossly approximates 
local time.  Thus working from a California location (zone = 8, or 
Pacific time) and attempting to get times printed as UT by giving a 
standard time zone as 0 will give peculiar behavior.  The {\tt g} option 
allows input in UT.
\par
As previously noted, daylight savings time is implemented using hard-coded 
algorithms to determine the dates on which the clock time changes.  If your 
location uses some different algorithm, you'll have to implement it in the 
source code.   Note that if you use daylight time and (as is the default)
specify your input times in local time, difficulties arise when daylight 
and standard times switch.  When daylight savings switches back to standard 
time (`fall back'), the numerical value of the time repeats for an hour; going the other 
way (`spring forward'), there is an hour of non-existent local times.  The program handles these 
conditions as follows.  If the specified time is within about 12 hours of 
the switch, a warning is printed.  If you specify a time during the hour 
which is skipped when daylight savings time begins, the computation is 
aborted and you are asked to specify a time 1 hour later.  If you specify 
a time during the double-valued period when the time drops back, the time 
defaults to standard and a rather sharper warning is printed.   This makes 
one hour of real time inaccessible (!) unless you switch to greenwich time 
input with the {\tt g} option and force the input time.
\par
The routine to turn JD into calendar date is adapted from {\it Astronomical
Formulae for Calculators}, Third Edition, (1985, Willman-Bell: Richmond).
The routine to generate the JD from 
the date and time was adapted from a routine originally based on a recipe 
in the old {\it American Ephemeris}.
\par
\begsubsect{\it Sun and Moon.}
\par
The lunar positions used are computed from Meeus' {\it Astronomical 
Formulae} book (op cit).
The routine corrects the time argument to approximate TDT, because the 
moon moves quickly enough to make these small timing differences significant.
Spot checks against the {\it Astronomical Almanac} indicate that the routine 
generates {\it geocentric} lunar positions good to better than a few 
seconds of time in RA and a fraction of a minute of arc in declination.  
A topocentric correction (from geocentric to observatory-centered, which 
can be $\sim 1^{\circ}$!) is included, based on an ellipsoidal earth and 
the true elevation of the observatory.  The topocentric correction appears 
to be somewhat more accurate than the lunar theory used.  

As noted earlier, under the {\tt =} command the program prints a notice if 
a solar or lunar eclipse is in progress.  The solar eclipse state is found 
very directly by computing the topocentric angular radii of the sun and moon 
and comparing with their topocentric angular separation.  The lunar eclipse 
calculation uses a simple geometrical model of the earth's shadow (taking 
into account the distance of the sun) at the moon's geocentric distance.  
Although the program does not generate eclipse timings directly, one can 
manually iterate to obtain times of eclipse contacts.  This provides an 
exacting test of the lunar ephemerides and the topocentric correction.  
For solar eclipses, the timings agree to within about 1 minute with the 
definitive ephemerides (e. g., F. Espenak and J. Anderson, NASA reference
publications Nos. 1301 and 1318, 1993); a 1 minute timing error implies 
$\sim 30''$ uncertainty in the moon's longitude.  Lunar eclipse contacts are
accurate to within about 5 minutes, with residual differences apparently 
due to the simple model used for the earth's shadow.  Thus these programs 
should not be used for the most critical eclipse calculations.

I do not know over what range of dates the lunar ephemeris can be expected 
to work well, but it works nicely toward the end of the twentieth century.

The printed phases of the moon are based on Meeus' algorithms, which he 
claims are good to $\pm 2$ minutes.
\par
Explicitly-printed positions of the sun are also from algorithms derived 
from Jean Meeus {\it Astronomical Formulae for Calculators}.  These positions 
are referred to the {\it mean} equinox of date.  A topocentric correction 
is applied (which amount to at most 8.8 arcsec).  Spot checks of the routine 
itself (modified for this purpose to show geocentric apparent rather than 
mean coordinates) gave agreement to a few arcseconds.  Rise/set times are 
derived using the {\it Astronomical Almanac} low-precision formulae for 
the sun, which are advertised as good to about $0.01^{\circ}$.
\par
If the observatory elevation above its horizon is specified as zero, the 
rising and setting times of the moon and sun are taken to be the times when
the center of the object is 50 arcmin below the geometrical horizon.  This 
is about the time of contact of the upper limb with the horizon, once 
refraction is taken into account.  Variations in the apparent diameter of 
the sun and moon are ignored.  If the observatory elevation above its 
horizon is non-zero, an approximate correction is added to the zenith 
angle at which rising and setting are reckoned; this is 
$$\hbox{horizon correction (radians)} = \sqrt{2 e \over R},$$
where $e$ is the observatory's elevation above its surroundings and $R$ 
is the radius of the earth.  In principle, a more accurate correction 
would simultaneously consider the effect of elevation on the refraction 
(although this doesn't make a great deal of difference -- see B. E. Schaefer 
and W. Liller, 1990, PASP, 102, 796, Table 4). In extreme cases (Mauna Kea!) 
the horizon correction can affect rise/set times by some 10 minutes.

Spot checks Schaefer and Liller's table of {\it observed} times of sunset 
for Mauna Kea and Cerro Tololo gave (for the most part) agreement to within 
about a minute; refraction variations preclude more accurate prediction.  
The observatory elevation above its surroundings is used only in the rise/set 
computations; the barycentric corrections and the topocentric correction 
for the moon use the observatory's elevation above sea level, which is a 
separate parameter.  Thus the elevation above the horizon may be adjusted 
to fit local circumstances.  The NOAO Newsletter tables for Kitt Peak, for 
instance, have sometimes included a correction of several hundred meters 
(smaller than the 2 km elevation of the observatory), the purpose being to 
correct approximately for the fact that Kitt Peak is higher than most of 
the mountains which define its horizon.

At very high latitudes, where the moon and sun graze the horizon, the 
program is less accurate since it iterates the rising, setting, and twilight 
times until the altitude of the object is within $0.1^{\circ}$ of the 
desired altitude. The rise and set algorithms are serviceable at circumpolar 
latitudes (see the section on geographical limitations below), but become
increasingly unreliable within a couple of degrees of the poles, where they 
are useless (at the poles, the diurnal rotation does not affect the altitudes 
of objects)!

The {\it lunar sky brightness} contribution is estimated if the sun is well 
down (beneath $-9$ degrees altitude) {\it and} both the moon and the object 
are in the sky.  This calculation follows K. Krisciunas and B. E. Schaefer 
(1991) PASP 103, 1033.  The calculation will not be even roughly accurate 
unless the sky is quite clear; haze, cloud, or even volcanic aerosols high
in the atmosphere can greatly affect the scattered moonlight!  To the 
Krisciunas and Schaefer model I've added a correction for variations in the 
apparent size of the moon and an extremely crude model of the `opposition 
effect', the surge of brightness just around full moon. This is modeled as a 
35 per cent brightening at full moon, which tapers linearly in phase and 
goes to zero at 7 degrees from full moon.  The code does check for 
{\it lunar eclipses}, but makes no attempt to account for their effect on 
the sky brightness.  It does print a disclaimer if the moon is in eclipse.
The brightness calculation assumes a zenith extinction of 0.172 mag in $V$,
typical of the 2800 m level on Mauna Kea.  Results are reported
as equivalent $V$ magnitude per square arcsecond; for comparison, 
the zenith night-sky brightness in a dark site is quite variable, but
is very roughly 21.5 mag per square arcsec in $V$.  
{\it These estimates should be useful for planning purposes, but unlike some 
of the other results in this program they are unlikely to be very precise.}

Similar cautions apply to the zenith twilight brightness.  This is
based on a polynomial fit to a graph on p. 38 of A. and M. Meinel's lovely 
book {\it Sunsets, Twilights, and Evening Skies} (Cambridge: 1983).  
Comparison with measurements by E. V. Ashburn, {\it Journ Geophys
Rsch}, v.57, p.85, 1952) shows that the fit provides a fair match
to the observed twilight in the {\it blue} (4400 \AA ); the $V$
band is about a magnitude fainter, and $I$ should be a little fainter
still.  The zero point of this number -- the dark night sky -- is 
quite problematic, but the dependence on the sun's zenith distance 
should be reasonably accurate.  Ashburn's data were taken from a 
California mountain site at an elevation of 1653 meters (5415 feet).  
 
\begsubsect{\it The Planets.}

The purpose of the planetary calculations is not to give definitive
positions (which are now derived from numerical integrations) but
to give rough positions for planning purposes ({\it e. g.,} is Jupiter 
visible?  Is it close to my object?).  If you really need to point 
blindly exactly at a planet, get another program or consult the 
{\it Astronomical Almanac!}

The positions are computed using formulae from the 1992 {\it Astronomical
Almanac} (p. E4).  The input data are heterogeneous.  For the planets 
through Mars, the program uses mean elements from the old 
{\it Explanatory Supplement\/} to the Nautical Almanac.  These give very 
good results (usually less than 1 arcmin) for the inferior planets and 
satisfactory results (a few arcmin) for Mars.  For the outer planets 
(Jupiter through Neptune), the input data are from Jean Meeus' 
{\it Astronomical Formulae for Calculators}, Third Editions (1985, 
Willman-Bell: Richmond).  The outer planets have such large mutual 
attractions that satisfactory positions can only be had by including a 
fair number of perturbation terms; I have included the largest ones from 
Meeus' Chapter 24.   The results are generally good to about 0.1 degree 
for Jupiter, and to a few tenths of a degree for Saturn, Uranus, and Neptune.
For Pluto, I have simply adopted the osculating elements for 1992.  These 
give very good positions for 1992, which slowly deteriorate the farther one 
gets from this date.

The planetary positions are used in two ways.  They can be printed out in a 
table using the option {\tt m}, which stands for ``major planets'' 
(``p'' is already used for proper motion).  More subtly, when one prints
circumstances using the {\tt =} command, the program computes the planetary 
positions and checks to see if your current RA and dec are within 3 degrees 
of any major planet.  If they are, it warns you.  The idea here is to avoid 
trying to observe some faint object with, say, Jupiter right next to it; 
the 3-degree tolerance was chosen as being about the radius of a Schmidt 
plate.  For asteroids, you're on your own!
\par
\begsubsect{\it Geographical limitations.}
\par
The daylight savings time conventions used are limited to those
which are coded. If you want to extend these to use at other sites 
you have to
code the new convention into the program and assign it a numerical code; 
negative numbers refer to southern
sites (daylight savings in, for instance December) and positive
to northern sites.  The routine to modify is called {\tt find\_dst\_bounds}.
\par
The algorithms used for rising and setting work at tropical
and temperate latitudes, and have been retrofitted to work
at very high latitude.  As noted above, rise/set times
are not as accurate at circumpolar latitudes as they are closer
to the equator, and they are meaningless at the geographical poles.  
The code has not been tested exhaustively at very high latitudes, nor has 
it been tested at length in the southern or eastern hemispheres, but there 
are no reasons for expecting it won't work there.  Problems with computation of
moonrise, etc., which should arise only at extreme latitudes,
are announced by a message reading 
\par
{\tt \qquad "Moonrise or -set calculation not converging!!".}
\par
These problems can arise because at very high latitudes, 
phenomena such as sunrise, twilight, and moonrise
do not always occur.  Thus the almanac section of the program tests
that each of these phenomena are likely to occur before 
attempting to compute when they do occur.  In this test,
it uses the declination of the relevant body computed for local
midnight; this can cause a mistake, especially
for the moon, which can change declination quickly.  This should
seldom be important.
\par
\begsubsect{\it Precession and apparent place.}
\par
The precession algorithm is coded from L. Taff's very 
useful book {\it Computational Spherical Astronomy} (Wiley).  It uses
a rotation matrix, which works correctly at the poles,
and gives mean positions good to much less than
1 arcsec in 50 years.
The set of test coordinates given by Smith et al. 
(1989, A. J. 97, 265) was reproduced to below 1 arcsec accuracy 
(except for proper motions near the poles).  The present version of the 
program uses IAU 1976 precession parameters.  It computes epochs as
Julian epochs, i.e., 
$\hbox{epoch} = 2000. + ({\rm JD - J2000}) / 365.25$.
\par
Hidden in the `extra goodies' menu is an {\it apparent place} routine,
which includes proper motion correction as above, nutation, and
aberration.  Because the routine prints at several stages of the
calculation, you can pick the desired level of effects (e.g, up to
but not including refraction).  
The nutation parameters are computed using approximate
series given by J. Meeus, {\it Astronomical Formulae for Calculators},
(Willman-Bell, 1985).  These include terms down to a few milli-arcsec
in amplitude.  The aberration uses a heliocentric (not barycentric) 
earth velocity derived from the accurate sun ephemeris, and neglects
relativistic terms and diurnal aberration.  
More importantly, {\it parallax is not
included} in the apparent place calculations, which for stars
leads to errors which are always less than $\sim 0.7$ arcsec
(and usually {\it much} less).  I spot-checked the program by 
computing the apparent places of five FK5 stars at times to match
the listings in  the 1994 
{\it Apparent Places of Fundamental Stars}. 
Agreement was generally $\sim 0.1$ arcsec and never much worse than
this, though none of the
test stars had large proper motions.  

If the object is above the
horizon, an estimate of the {\it refracted} position
is printed.  This uses formulae adopted from the {\it Astromomical Almanac}
for 1995, p. B62.  The air temperature is assumed to be 
20 Celsius, and the 
atmospheric pressure is computed using an exponential
atmosphere and the elevation of the site above sea level.  The refracted
position is printed even if the object is below the geometrical 
horizon, provided refraction is estimated to bring it above the 
level-surface horizon (depression of the horizon due to elevation
above surroundings is not included in this calculation). 
The refraction correction in general
cannot be expected to be very accurate, but it's better than nothing!

Note that because of refinements in the reference frame, 
proper motions should in principle be transformed at the same time as
coordinates; the current
program ignores this.
\par
\begsubsect{\it  Local Mean Sidereal Time.}
\par
Strictly speaking, the local sidereal time equals the hour angle of the 
vernal equinox; the {\it mean} sidereal time computed here is slightly 
different, because the effect of nutation on the location of the equinox 
is not included.  This correction is called the `equation of the equinoxes', 
which is tabulated in the {\it Astronomical Almanac}; it's generally 
less than 2 sec. The algorithm used here is based on formulae and 
procedures explained in the 1992  {\it Astronomical Almanac}, pp. B7 and L2.
Tests for the longitude of Greenwich in 1992 give agreement with the 
Astronomical Almanac tables of {\it mean} sidereal time to within a few msec. 
Also, the IRAF routine gives the same answers to within 0.1 sec.  However,
strictly speaking this accuracy will obtain only if your input time is based 
on the correct type of UT; UTC (broadcast by WWV) is tied to atomic time
and is corrected by whole seconds to agree with earth rotation.  If your 
input is based on UTC (as all civil time is), the computed local mean 
sidereal time will be incorrect by UT $-$ UTC, which is less than 1 second.
\par
\begsubsect{\it Parallactic Angle.}
\par
This quantity -- the position angle of a great circle connecting the
object to the zenith -- is often used for setting a spectrograph
slit along the angle of atmospheric dispersion.  Its name arises because
it is the position angle along which a very nearby object (such as the
moon) will be displaced by topocentric parallax.  Its importance for
spectrophotometry is emphasized and quantified
by A. Filippenko (1982, PASP, 94, 715).  He gives formulae 
for the parallactic angle, but leaves unclear a 
choice of root of an inverse trig function; my own routine 
computes the altitude and azimuth, which allows an unambiguous choice 
of root since more elements of the spherical triangle are known.
Because some applications (such as spectrograph slit angles) are 
indifferent to 180-degree rotations of this quantity, the {\tt =} option
also prints the antiparallel angle (parallactic $\pm 180$) in square brackets.
\par
\begsubsect{\it Airmass.}
\par
The airmass as such is only reported by the instantaneous circumstances
option {\tt =}; elsewhere the secant of the zenith angle, secant $z$,
is given.  For zenith distances less than 60 degrees, these two quantities
are equal to better than 1 per cent, while at large distances they
diverge somewhat.  The airmass used is based on a polynomial fit to 
a table given by C. M. Snell and A. M. Heiser in PASP, vol. 80, 0. 336 (1968).
Their table is calculated for a standard atmosphere and the elevation
of Kitt Peak.  The fit is
$$\sec z - \hbox{airmass} = 2.87947 \times 10^{-3}\ y +  3.03310 \times 10^{-3}\ y^2 + 1.35117 \times 10^{-3}\ y^3 - 4.717 \times 10^{-5}\ y^4, $$
where
$$ y = \sec z - 1.$$
A small constant term in the fit is suppressed to force the airmass to 
unity at the zenith.  The fit is constrained for $z \le 85$ degrees, and
passes smoothly through the tabulated points to within $\pm 4 \times 10^{-4}$, 
which is probably smaller than natural variation due to atmospheric
conditions.  I've heard anecdotally that the airmass approaches about
35 toward the horizon, while $\sec z$ tends toward infinity, so this
approach clearly is limited to $z \le 85$ degrees; beyond that, 
the value of $\sec z$ is printed.
\par
\begsubsect{\it Barycentric (`Heliocentric') Corrections}

The algorithms used for the earth's orbit are derived from the solar 
ephemeris, which in turn is from Jean Meeus' 
{\it Astronomical Formulae for Calculators}, pp. 79$\it ff$.  It uses 
an elliptical earth orbit and a few of the most important perturbations.  
The correction to the solar system barycenter is also included, using the same
planetary calculations discussed earlier.  
The earth's diurnal rotation (assuming an ellipsoidal earth and 
including the observatory's elevation) 
is included in the velocity calculation, but the time-of-flight across
the earth's radius ($\sim$ 0.02 sec) is not included.
Meeus' solar theory does include a rough correction for the recoil of
the earth due to the moon.
I tested these against the JPL DE200 ephemeris (the basis for the 
{\it Astronomical Almanac} tables) at 10-day intervals for 
2000 days; the maximum errors were 0.11 s and 3.02 m/s,
and the RMS errors were 0.058 sec and 1.8 m/s.
The most demanding applications, such as analyses of pulsar
timing and doppler-based searches for extrasolar planets, will require
better accuracy, but this should be adequate for almost everyone else.

\begsubsect{\it Galactic and Ecliptic Coordinates.}
\par
The galactic coordinates conform strictly to the IAU definition and
agree closely with those computed by IRAF; they are based on a
rotation matrix and do not suffer ambiguities due to the roots of
inverse trig functions.  
The input coordinates are precessed to 1950 before being 
transformed to galactic, which introduces a slight 
uncertainty.  If 1950 input coordinates are supplied,
the only source of error should be double-precision roundoff!
The ecliptic coordinates should be good to $< 0.001$ degrees.

\begsubsect{\it BUGS and other ungraceful behavior.}

I actually don't know of any real bugs (!), but the program
can behave in a peculiar fashion given some inputs.

If you depart from the specified input formats, you can get 
peculiar behavior.  Some error checking is done, and some prompting
is given in some cases where really unexpected input is found, but
these routines are less than perfect.  It is difficult
to crash the program, but it can still be forced into an 
infinite loop in which it asks for a valid date -- at this point,
you'll just have to crash it and start over.  If you can 
document the inputs which cause this behavior, I'd appreciate
if you would drop me a line so I can fix the problem.

The specification of times and dates is a little ungraceful (see the
discussions of the {\tt g} and {\tt n} options above); the
`night date' option patches one potentially confusing condition with another.
To some extent these confusions are inevitable; astronomers are forced to
work at night, when dates change to suit the convenience of
everyone else!  The definition of JD, with its infernal half-step
difference against UT dates, is a historical example of an ill-considered
attempt to get around these difficulties.  

In former versions, the day and date could in principle disagree 
within a very close tolerance of midnight.  I believe I have
eliminated the possibility of this by using the same truncation
of the Julian date to derive the day and the date.

The conversion from local to UT is tricky around
the time when daylight savings time changes to standard
and {\it vice versa}.  The behavior has been rationalized in 
recent versions, but it's still tricky.  The hour when 
daylight time changes back to standard time is ambiguous -- there
is a default to standard time which may not be what the user wants.  
The user is warned if
there might be a problem.  Conversion from UT to local appears
to be rigorously correct, so specifying times as UT when there
is a problem should get around any difficulties.

After you type {\tt g} to toggle between greenwich and local
time, the time currently in effect changes to the value
which is {\it numerically} the same in the new system, not the
time which is actually equivalent.  So if you are in the zone
7 hr west (Mountain), and you are using local time, and
the time is 1991 Jul 7, 22 hr 0 mn 0 sec MST; and you type {\tt g},
the time in force is now 1991 Jul 7, 22 hr 0 mn 0 sec {\it Universal time},
which is 7 hr {\it earlier}.  To get the same actual time, you'd have to
enter y 1991 7 8 and t 5 0 0.  The program reminds the user
that this is happening.  Similar considerations apply to the
{\tt n} option.

Similarly, typing {\tt e} to change the epoch assumed for the input
coordinates doesn't precess any actual coordinates; it just changes
how input will be interpreted.

On these points, it may
be helpful to emphasize that the program doesn't actually calculate or
convert anything until it's asked for output -- the numbers you type
in lie dormant until then.  Thus the {\tt g}, {\tt n}, and {\tt e} options
control only the {\it interpretations} of your input parameters.

The twilight and rise/set times are slightly inaccurate at
very high latitudes, since the object comes into the appropriate
altitude at a grazing angle.  Rise/set can be erroneously
reported as not occurring at very high latitudes because the
occurrance of rise/set is judged using the position for local
midnight, and it's possible in principle for the program to 
try to find a rise/set time which actually doesn't occur.

The correction used for the site elevation in the rise/set
calculations is approximate.  Note that 
it may or may not be appropriate to include altitude corrections
for your site, based on the details of your local horizon.

\begsubsect{Notes for Programmers.}

The {\tt skycalc} calculator program 
allows the user to turn on a {\it log file}, so that one can
save results without having to go through the rigamarole of
creating an input file and redirecting output.  I implemented this
by replacing the appropriate calls to {\tt printf} with a new routine,
{\tt oprntf}, which optionally writes to a file.  Because
{\tt printf}, and hence {\tt oprntf}, have variable numbers of arguments, 
I used the widely-available ANSI-standard framework (which involves
the inclusion of {\tt <stdarg.h>}; see 
Kernighan and Ritchie, 2nd edition, p. 155 {\tt ff}).  It turns out that
the {\tt cc} compiler on older Sun workstations does not support this
standard; the supposedly ANSI-standard version {\tt acc} has
a compiler bug affecting this particular feature (I'm indebted to
Mike Fitz of NOAO for chasing this down).  However, the Open
Software Foundation's {\tt gcc} compiles this correctly.  I'd
urge users of Sun machines to acquire this compiler.

It's not always appropriate for the user to have write permission.
Accordingly it's possible to recompile the program with such
permissions turned off; to do this, find the preprocesser definition 

{\tt \#define LOG\_FILES\_OK 1} 

and change the {\tt 1} to a {\tt 0}.

I encourage programmers to borrow from this code, and to modify it
for their purposes (especially by adding sites).  However, it's
probably not a good idea to attempt major surgery on the code itself unless
you study it for a while.   There are some subtle issues involved
which took me a while to get right.  But most minor
changes can be done safely and rather trivially.

If you find that the routines which depend on the system clock
don't work, you can turn them off by finding the line

{\tt \#define SYS\_CLOCK\_OK 1}

and changing the {\tt 1} to any other number.

I've run into one curious issue regarding system clocks.  At Lick
Observatory, and perhaps others, there is a tradition of maintaining 
standard time all year for scientific purposes, even though civil
timekeeping uses daylight savings time.  It's tempting to implement 
this simply by turning off daylight savings, but then the {\tt T}
option doesn't work for half the year, because the computer's clock
is generally set to civil time.  I haven't yet coded in the 
complications needed to patch this.

To include a new site on the menu, modify the routine
{\tt load\_site}.  Follow the examples there.  You can check
that they are entered correctly by examining output from 
the {\tt l} option in the program.  Note carefully that the 
program expects longitude as west longitude in units of hours, 
minutes, and seconds, where 1 h = 15 degrees.

If your site uses an unsupported daylight savings time algorithm,
include your option in the routine {\tt find\_dst\_bounds}, using
the current routine as a model.  Note that in the labels for
dst conventions, positive numbers refer to northern sites and
negative to southern sites, for which the logic has to be reversed.

The code at this point is packaged as a single enormous source
file, of over 7000 lines!  Function declarations are mostly in the
`old' K\&R style.  I also have a rather older version in which the
code is sliced into eight cross-referenced pieces, with 
ANSI-style function declarations.  If you need this (say to
compile on a PC), send me email.  The older version is without 
many of the newer bells and whistles.

Previous versions of the code included some widely published 
functions (for sorting and JD conversion) for which copyright was 
claimed by others.  In the 2000 version, these have been  
replaced with non-proprietary functions; the JD conversions are
an original coding of a Jean Meeus algorithm, and the sort routine is
coded from a heapsort algorithm published by Ken Iverson
with the Association for Computing Machinery. As far as 
legalities are concerned, the code should now be pure
as the driven snow.


\par\vfill\eject

\centerline{\bf 2. A NIGHTTIME ASTRONOMICAL CALENDAR PROGRAM.}
\par                     
This program prints an astronomical calendar for a given year
from a single site.  The algorithms used are for the most part
identical to those used for the circumstances calculator program
described above, but the input and output are different.
The program has been used for several years to print the 
nighttime astronomical calendar for Kitt Peak included in the
NOAO Newsletter, and is used to generate the calendars 
at a number of observatory websites.  Again, this is a 
self-contained C program which should run gracefully on various computers;
however, the cautions listed above apply.

The output has a wide format (122 characters).  At the beginning,
some information is printed along with prompts for interactive use
(something which will probably seldom happen.)  Then comes a page
of information about the program and its accuracy, which
is largely redundant with this document.  Next follows a table
of moon phases, with the times, given as local zone times,
accurate to a few minutes.  Next follow the results
for each of the twelve months.  

The output may optionally be formatted for input into the TeX
\footnote{*}{TeX is probably someone's trademark, or something.}
typesetting program.  It can be set up to print two months to 
the page in portrait orientation, or one month to the page in
landscape orientation.  Further details on the TeX option are 
given later.  If TeX output is not selected, a formfeed character 
is inserted at the top of each page.

At the head 
of each month is the year and month set off by asterisks.
Also given is the site name, its longitude in {\it hours} minutes and seconds,
its latitude in {\it degrees} and decimal minutes, and the standard time
zone.  After some other information, the user is reminded that the 
times listed (except for sidereal) are local zone times; the name of the
zone is given.  If daylight savings time is used, the user is reminded
of this as well.

The rest of the calendar is organized with one night per line.
Note that this choice is only sensible for nighttime astronomers,
a large (but not all-inclusive) subset.  Though the calendar works
at circumpolar latitudes, this form of organization is not optimal
during the ``midnight sun'' either!  A detailed description of the
tabulated quantities follows.

The first column gives the day and date, {\it for both evening and morning}.
This should minimize errors in reading dates.  A blank line 
appears between Saturday and Sunday nights.

The next column gives the JD at {\it local} midnight, rounded off to
the nearest 0.1 d to avoid any ambiguity.  The number given has
the largest multiple of 10000 days (figured for the first of the
month) subtracted away; thus JD 2451020.5 will be printed as 1020.5.
If daylight savings 
is in use, the JD is the JD of daylight savings midnight.

The third column gives the Local {\it Mean} Sidereal Time (see
the earlier discussion for the distinction between true and
mean sidereal time) at local
midnight; it is more accurate
than the nearest-second accuracy which is tabulated.  Again, 
if daylight savings is in effect, this is the JD at daylight-savings
midnight.

The next four columns give respectively the times of sunset,
evening (18-degree) twilight, morning twilight, and sunrise.  Thus the columns 
are in the same sequence as the actual events, which seems
natural.
The twilight given is 18 degree (`astronomical') twilight, 
and the rise/set times given are when the center of the sun 
is 50 arcminutes below the horizon; this is roughly the time 
when the sun's upper limb touches the horizon, once refraction
and the sun's angular diameter are taken into account.  If the
observatory's elevation above its surroundings are specified, 
the depression of the horizon is taken into account.  Only 
a single number is used for observatory elevation above the
surroundings; there's no attempt to account for the topography
of the horizon.  Accuracy
is as discussed above.  If an event doesn't happen during a night
({\it e.g.}, twilight at high latitudes in summer), ellipses ($\ldots$)
are printed in the appropriate column.  

The next two columns give a very useful quantity, namely
the sidereal times at evening and morning twilight.  This defines
the range of RA which is accessible on the meridian during the night.

The last five columns pertain to the {\it moon}.  

The times of moonrise and moonset are given, provided they 
occur at night or within a boundary on either side.  (The algorithm
used here can give some trouble if the site is very far from the
center of its assumed time zone.)
Ellipses ($\ldots$) are printed if
the rise or set does not occur within the specified interval.  Note
that rise and set times, though they occur in successive columns,
do {\it not} always occur successively in time, depending on the moon's
phase.  Rise/set times are again for 50 arcminutes below the horizon;
variations in the moon's semidiameter are ignored.  The lunar ephemerides 
are based on accurate formulae from Jean Meeus' {\it Astronomical
Formulae for Calculators}.

The next column gives the percentage of the moon's face which is
illuminated.  New, first quarter, full, and last quarter
correspond approximately to values of 0, 50, 100, and 50
respectively for this quantity.  If $\theta$ is the angle subtended by
the sun and the moon at the observer, the quantity tabulated is

$$\hbox{Illuminated percentage} = 100. \times {1 \over 2}(1 - \cos \theta ).$$

Finally, the last two columns give the RA and dec of the moon at local
midnight, whether or not the moon is up at that time.  The position
is topocentric.  It is very useful to know the moon's position 
if you're trying to work around it.

\begsubsect{Times in the Calendar program.}

Note that the rise, set, and twilight times given in the calendar 
are for {\it the local time zone}.  (The sidereal times are of course 
strictly local and have nothing to do with the time zone.)  If 
you wish, daylight savings times can be listed; if you use
this feature, a site-dependent prescription is used to select
whether daylight or standard time is used.  The
switchover occurs at 2 AM on Sunday morning.  This can in principle
lead to an ambiguity around the time of time change (in the fall,
it's 1:30 AM local time twice on the same night!), but you should
be able to unravel this rare case from continuity with the
preceding and following nights.  Several conventions are
available for the dates of the time change.
The conventions coded are the USA convention (1st
Sunday in April to last in October after 1986, last Sunday in April
before 1986, more or less), the Chilean convention, the 
Spanish convention, and the Australian convention.  If you need 
another convention you'll have to
add it to the source code, in the routine {\tt find\_dst\_bounds}.

The header that appears on each page makes a note if daylight savings
time is used.  An asterisk is printed by the date on which the
time is changed.

\begsubsect{Running the calendar.}

You will probably never want to actually run the calendar interactively; 
it takes a while to run, and it produces a very wide output. 
It's more appropriate to run it in background with
output redirected to a file you can print on some wide device
({\it e.g.}, a laserwriter in landscape mode).

I describe below the input that the program will call for when run
non-interactively.  However, the program is also designed so that you
can `test-run' it interactively to reconnoiter the inputs it will
require and the options available.  The program first asks you
to select a site, and prints a menu of `canned' possibilities.
You can select one of the `canned' sites, or enter all the parameters
for another site.  The last input prompted for is the year for which
to print the calendar; giving a negative year here exits the program.

Before producing the calendar, then, your first step should be
to run the program interactively, mostly to be sure which
`canned sites' are available in your own version of the program.
After that, you create a little procedure or job, with the
output redirected into a file, to make the calendar itself.
The exact format of this job will depend on your system, and
on just what you want to do, but the sequence of inputs you need is 
system-independent.  Here are some annotated examples; the
text to the right is commentary, not to be put in 
the job itself.

Example 1 - for one of the `canned sites.'.
\parskip 0pt plus 0.5pt minus 0pt
\verbatim$

k             (code for kitt peak, assumed to be 
                             one of the "canned sites")
2             (do format for TeX printing, 2 months per page.)
2002          (year for which calendar is to be run)
$
\parskip 2ex plus 0.5ex minus 0.5ex

Example 2 - for another site.

\parskip 0pt plus 0.5pt minus 0pt
\verbatim$

n               (new site - not one of the canned ones).
6 16 56	        (WEST longitude, HOURS MINUTES SECONDS.)
44 44 42        (latitude, DEGREES MINUTES SECONDS).
0               (site elevation, in meters, above effective horizon)
6               (standard time zone, hours WEST of Greenwich)
USND Hoople     (Site name; terminate with carriage return).
Central         (Time zone name, terminated with carriage return).
1               (use daylight savings time, 
                          USA post-1986 prescription).
0               (don't use TeX on output -- use 1, 2, or 3 for TeX).
1991            (year)
$
\parskip 2ex plus 0.5ex minus 0.5ex


Naturally, you should be especially careful about your site parameters;
anyone entering a new site in the source code should be downright
compulsive, since many people may depend on the accuracy.  The user
should check the  output to be sure the parameters repeated are correct; 
the latitude, longitude, etc. are printed at the top of every month's page.

\begsubsect{Examples of how to run the program in background.}

I'll show how to do this using UNIX or VMS systems (note that these
are trademark names).
This is not of complete generality, but covers the bases for most
users.

On a UNIX system, if you've named your executable task {\tt calendar},
you'd edit a file called {\tt inputs} containing the input data,
just as above.  Then type 
\par

{\tt \qquad calendar < inputs > hoople\_1991 \&}

\par
which could be paraphrased as ``run the calendar program, taking 
input from ({\tt <}) the file named {\tt inputs}, 
directing output to ({\tt >}) a 
file named {\tt hoople\_1991}, and do it in background ({\tt \&})''. 

On a VMS system, you would edit a command file -- let's call
it {\tt CALDRIVE.COM} -- which would have the first line
\par
{\tt \qquad \$ run calendar}
\par
with subsequent input on successive lines without dollar signs at the
front (again, just as above).  Then you'd type 
\par
{\tt \qquad @CALDRIVE/OUT=HOOPLE\_1991.LIS}
\par
to run the command file and direct the outfile to a file called 
{\tt HOOPLE\_1991.LIS}.

\begsubsect{More on the TeX option.}

The TeX output is based on a `Dirty Trick' given by Donald Knuth in
{\it The TeXbook} on page 382; his macro {\tt verbatim} 
simply prints a section of text using the {\tt tt} font, which has
a fixed character width.  If you select the TeX option, 
you'll have to edit your redirected output file to
remove the `fossil prompts' which come at the beginning;
look for the line marked {\tt CUT HERE}.  The edited file
{\it should} then set up and print normally.  If you
used option 1, you'll get one month on
each page in portrait mode, option 2 gives two months on 
one page in portrait mode, and option 3 gives one month on
each page in landscape mode. You'll also get a cover page 
and the moon-phase table.  

Note that there are several
parameters right at the start of the TeX file which may need to
be `tweaked' to your local printer.  
The portrait mode version sets {\tt $\backslash$magnification 835,
$\backslash$hsize 7.6 truein}, and {\tt  $\backslash$hoffset -0.7 truein}.  
These parameter values simply
make the very wide output as large 
as possible on our local system.   Because TeX defaults to a magnification
of 1000, the value 835 makes the print fairly small.
One can always play with these parameters to make the output come
out nicely on your local printer.
You may wish to make the {\tt  $\backslash$baselineskip} a little larger if you
find the lines to be too crowded vertically; increasing it to
12 spreads things out a lot. 

\begsubsect{Sample Output from the Calendar Program.}

The following output came directly from a run of the program using
the input quantities given in the example above.  It should be used
to check the results in a new site.  

Because the output is so wide, I've added carriage returns; in the 
main body of the calendar, they are consistently right after the
columns relating to the sun.  This has a terrible effect on legibility, 
but fits it on the page $\ldots$
\parskip 0pt plus 0.5pt minus 0pt
\verbatim$

                                        ***** 1991 JANUARY *****

Calendar for Univ. South. North Dakota at Hoople, west longitude 
(h.m.s) = 6 16 56, latitude (d.m) =  44 44.7
Note that each line lists events of one night, spanning two calendar 
dates.  Rise/set times are given
in Central time (  6 hr W), uncorrected for elevation, DAYLIGHT time 
used, * shows night clocks are reset.
Moon coords. and illum. are for local midnight, even if moon is down.  
Program: John Thorstensen, Dartmouth College.

  Date (eve/morn)      JDmid    LMSTmidn   ---------- Sun: ---------   
LST twilight:  ------------- Moon: --------------
  (1991 at start)    (-2440000)            set  twi.end twi.beg rise
    eve    morn    rise   set  %illum   RA      Dec

Tue Jan 01/Wed Jan 02  8258.8    6 28 34   16 47  18 33   6 09  7 54
    1 00  12 38   18 08  .....    97   8 17.7  18 56
Wed Jan 02/Thu Jan 03  8259.8    6 32 31   16 48  18 33   6 09  7 54
    1 05  12 42   19 29  .....    91   9 15.5  13 45
Thu Jan 03/Fri Jan 04  8260.8    6 36 28   16 49  18 34   6 09  7 54
    1 10  12 46   20 47  .....    83  10 08.9   7 58
$
\parskip 2ex plus 0.5ex minus 0.5ex
\bigskip 	
\centerline{\bf 3. CAUTIONS APPLYING TO BOTH PROGRAMS.}

When these codes are ported to a new system, the results should 
be checked carefully for accuracy.  The sample output 
in this document should be reproduced correctly.  The user 
assumes responsibility for the correct operation of the 
programs and the sensible interpretation of their results.  
The user's attention is drawn to the known limitations
of the algorithms documented above.

While the programs have been tested carefully, with
the results given above, the author makes no guarantee that this
level of accuracy will obtain in all circumstances on all machines.
I explicitly disavow any responsibility, express or implied, for
damages resulting from use of the program.  Output from 
this program should never be used as evidence in a court of law,
or to make decisions which might cause bodily harm if the
results weren't right.


\begsubsect Miscellany

  All the source code is usually kept in a single file,
which contains all the subroutines as well as the main program.
The size can cause difficulty with some compilers on personal computers 
(though the {\tt gcc} compiler which is
standard in Linux takes it in stride).  It's possible to break the code
up by using function prototypes in the usual manner, but
it's a lot of labor.

\begsubsect{Maintainance.}

  If you find a real problem, not due to your local machine and
not documented above, write

\par\obeylines\obeyspaces
	John Thorstensen
	Dept. of Physics and Astronomy
	Dartmouth College
	Hanover, NH 03755

	John.Thorstensen@dartmouth.edu

\par\vfill\supereject\end
